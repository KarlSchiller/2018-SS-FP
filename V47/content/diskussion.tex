\section{Diskussion}
\label{sec:Diskussion}

% Besprechung von Cv
% Berechnung der Debye-Temperatur
Mittels der Formeln % TODO \eqref{} Formeln 7 und 10
lassen sich die Debye-Frequenz und die Debye-Temperatur theoretisch
berechnen. Unter Verwendung von
$v_\text{trans}=\SI{2.26}{\kilo\meter\per\second}$,
$v_\text{long}=\SI{4.7}{\kilo\meter\per\second}$ \cite[p.~5]{anleitung}
und $N_\text{L}=\SI{2.6516467(15)e25}{\raiseto{-3}\meter}$ \cite{Codata}
ergibt sich
\begin{align*}
    % TODO
    % Theoretische Werte für Debye Frequenz und Temperatur
\end{align*}

% TODO
% Vergleich mit Literaturwert Debye-Temperatur

% Diskussion der Abweichungen:
- Cv konstant zu klein, da Zylinder immer etwas wärmer und somit Wärmestrahlung
- Konvektion und Wärmeleitung vllt. nicht vollständig verhindert
