\section{Diskussion}
\label{sec:Diskussion}

% Besprechung von Cv
Leider spiegeln die experimentell ermittelten $C_\text{V}$-Werte den
theoretisch vorhergesagten Zusammenhang nur sehr begrenzt wieder, siehe
dazu Abbildung \ref{fig:cv}. Im Bereich bis ungefähr \SI{100}{\grd} sollte sich
ein Verlauf ähnlich zu $C_\text{V} \sim T^3$ einstellen, bei höheren
Temperaturen sollte die Wärmekapazität gegen einen konstanten Wert
streben. Es zeigt sich, dass in beiden Bereichen die experimentell
ermittelte Wärmekapazität zu gering ausfällt, was ein Indiz für einen
systematischen Fehler sein könnte. Dabei ist anzumerken, dass aufgrund
schwieriger Synchronisation der Erwärmungen von Probe und Zylinder
der Zylinder immer wärmer war als die Probe. Es ist also zu einer
Wärmestrahlung seitens des Zylinders an die Probe gekommen, welche die benötigte
Heizleistung für eine Temperaturerhöhung erniedrigt und somit für kleinere
experimentelle $C_\text{V}$-Werte verantwortlich ist.

% Besprechung der Debye-Temperatur
Bei den ersten zwei Messwerten trat eine Temperaturerhöhung auf, die
größer als \SI{11}{\celsius} war, sodass hier die Pt-100 Elemente nicht
genau dem angenommenen Zusammenhang \eqref{eqn:TausR} folgen. Bei den Messwerten
\numrange{4}{6} war die Zylindertemperatur deutlich größer als die
Temperatur der Probe, sodass hier die Wärmekapazitätswerte deutlich kleiner
ausfallen. Da jedoch nur die ersten \num{8} Messwerte zur Berechnung der
Debye-Temperatur verwendet wurden, lässt sich hier eine starke Abweichung
zum Literaturwert festellen:
\begin{align*}
    \theta_\text{D,exp} &= \SI{915.9(4)}{\kelvin} \\
    \theta_\text{D,lit} &= \SI{347}{\kelvin} \text{ \cite{door} bei }T = \SI{0}{\kelvin} \\
    \theta_\text{D,lit} &= \SI{310}{\kelvin} \text{ \cite{door} bei }T = \SI{300}{\kelvin}
\end{align*}

% Berechnung der Debye-Temperatur
Mittels der Formeln \eqref{eqn:debye-temperatur} % TODO \eqref{} Formeln 7 und 10
lassen sich die Debye-Frequenz und die Debye-Temperatur theoretisch
berechnen. Unter Verwendung von
$v_\text{trans}=\SI{2.26}{\kilo\meter\per\second}$,
$v_\text{long}=\SI{4.7}{\kilo\meter\per\second}$ \cite[p.~5]{anleitung}
und $N_\text{L}=\SI{2.6516467(15)e25}{\raiseto{-3}\meter}$ \cite{Codata}
ergibt sich
\begin{align*}
    \omega_\text{D,theo} &= \SI{8.7640029(17)e+13}{\raiseto{-1}\second} \\
    \theta_\text{D,theo} &= \SI{669.41515(13)}{\kelvin}
\end{align*}
Hier weicht der Wert von dem experimentell Bestimmten um \SI{26.9}{\percent}
nach oben ab, was sich jedoch mit den oben genannten Fehlern erklären lässt.
Der theoretisch berechnete Wert ist etwa doppelt so groß wie der Literaturwert.
Dies könnte daran liegen, dass die Probe als würfelförmig angenommen wurde und
ihr tatsächliches Volumen etwas abweicht.

% Nochmalige Auswertung mit Anhebung der Cv-Werte
Als Überprüfung, ob der oben genannte systematische Fehler tatsächlich für
die Abweichung der Debyetemperatur verantwortlich sein kann, sind die
Messwerte für $C_\text{V}$ in Tabelle \ref{tab:neu} bis \SI{170}{\kelvin}
dargestellt. Dabei wurden sie jedoch um \SI{16}{\joule\per\mole\grd}
angehoben und aus den oben genannten Gründen wurden die ersten zwei Messwerte
weggelassen. Abermals wurden in einer Tabelle in der Versuchsanleitung
\cite[p.~5]{anleitung} die zugehörigen Werte von $\sfrac{\theta_\text{D}}{T}$
nachgeschlagen und mit $T_\text{mittel}$ multipliziert. Zusammen mit den
sich ergebenden Debyetemperaturen sind sie ebenfalls in Tabelle \ref{tab:neu}
aufgelistet. Bei diesen höheren $C_\text{V}$-Werten beträgt die
Debye-Temperatur im Mittel \SI{316.45(15)}{\kelvin} und passt sehr gut zu
dem Literaturwert.

\begin{table}
  \centering
  \caption{Optimierte Messwerte zur Berechnung von $\theta_\text{D}$.}
  \label{tab:neu}
  \sisetup{table-format=1.1}
  \begin{tabular}{S[table-format=3.1] @{${}\pm{}$} S
                  S[table-format=2.2] @{${}\pm{}$} S[table-format=1.2]
                  S
                  S[table-format=3.1] @{${}\pm{}$} S}
  \toprule
      \multicolumn{2}{c}{$T_\text{mittel} \: / \: \si{\grd}$} &
      \multicolumn{2}{c}{$C_\text{V} \: / \: \si{\joule\raiseto{-1}\mole\raiseto{-1}\grd}$} &
      {$\sfrac{\theta_\text{D}}{T}$} &
      \multicolumn{2}{c}{$\theta_\text{D,neu} \: / \: \si{\celsius}$} \\
  \midrule
  -161.4 & 0.2 & 23.87 & 0.22 & 0.9 & -172.5 & 0.2 \\
  -138.5 & 0.2 & 19.12 & 0.05 & 2.4 & 50.1 & 0.4 \\
  -120.7 & 0.2 & 18.40 & 0.04 & 2.5 & 108.0 & 0.4 \\
  -113.7 & 0.2 & 18.72 & 0.04 & 2.5 & 125.6 & 0.4 \\
  -109.6 & 0.2 & 19.54 & 0.05 & 2.3 & 103.1 & 0.4 \\
  -105.4 & 0.2 & 21.06 & 0.09 & 1.9 & 45.5 & 0.3 \\
\end{tabular}
\end{table}

