\section{Zielsetzung}
\label{sec:Zielsetzung}
In diesem Versuch soll die molare Wärme von kristallinem Kupfer bestimmt werden.
Dazu wird die Debye-Temperatur $\theta_\text{D}$ experimentell bestimmt und mit einem theoretischen
Wert verglichen.

\section{Theorie}
\label{sec:Theorie}

\subsection{Klassische Theorie}
\label{sec:Klassische Theorie}
In der klassischen Theorie der Molwärme können die Atome aufgrund der Gitterkräfte nur
in aufeinander senkrechten Raumrichtungen schwingen. Nach dem Äquipatitionstheorem
gilt für die mittlere Energie pro Atom
\begin{equation}
  \langle \text{u} \rangle = 3\,\text{k}\,T
\end{equation}
Für ein Mol beträgt der Wert also
\begin{equation}
  \text{U} = 3\, k\, \text{N}_\text{L}\, T = 3\,\text{R}\,T,
\end{equation}
wobei $N_\text{L}$ die Loschmidtsche Zahl und $\text{R}$ die allgemeine Gaskonstante ist.
Bei einem konstanten Volumen führt dies zu einer spezifischen Molwärme von
\begin{equation}
  C_\text{V} = 3\,\text{R}.
\end{equation}
Dies beschreibt bei hohen Temperaturen das Dulong-Petitsche Gesetz und ist weder
Temperatur- noch Materialabhängig.

\subsection{Einstein-Modell}
\label{sec:Einstein-Modell}
Durch eine Quantelung der Schwingungsenergie mit konstanter Kreisfrequenz $\omega$
können Energien von n$\hbar\omega$ (n $\in \mathds{N}$) aufgenommen und abgegeben werden.
Die mittlere Energie lässt sich nun durch eine relativen Wahrscheinlichkeit W(n)
ausdrücken, die der Boltzmann-Verteilung folgt:
\begin{equation}
  \text{W}(\text{n}) = \exp{\left(-\frac{\text{n}\hbar\omega}{kT}\right)}
\end{equation}
Die mittlere Energie nach Einstein ergibt sich durch Abzählen und Normieren aller möglichen Energien. Daraus ergibt sich:
\begin{equation}
  \langle \text{u} \rangle_\text{Einstein} = \frac{\hbar\omega}{\exp{\left(\frac{\hbar\omega}{\text{k}T}\right)}-1}
\end{equation}
Der Ausdruckfür die molare Wärme wird als Einstein-Funktion bezeichnet
und liefert für hohe Temperaturen wieder ein asymptotisches Verhalten gegen 3R:
\begin{equation}
  \text{C}_\text{V} = 3\,\text{R}\,\frac{\hbar^2\omega^2}{\text{k}^2T^2}\frac{\exp{\left(\frac{\hbar\omega}{\text{k}T}\right)}}{(\exp{\left(\frac{\hbar\omega}{\text{k}T}\right)}-1)^2}
\end{equation}

\subsection{Debye-Modell}
\label{sec:Debye-Modell}
Im Debye-Modell wird die Einstein-Frequenz durch eine spektrale Verteilung Z($\omega$)
ersetzt. Diese Verteilung beinhaltet alle auftretenden Eigenschwingungen der Atome
und kann bei Festkörpern
beliebig kompliziert werden. Als Annahme wird getroffen, das die Phasengeschwindigkeit
weder von der Frequenz, noch von der Ausbreitungsrichtung abhängt. Somit können in
einem Festkörper mit Seitenlänge L in einem Intervall von $\omega$ bis $\omega + \text{d}\omega$
die Eigenschwingungen abgezählt werden. Daraus ergibt sich:
\begin{equation}
  \text{Z}(\omega) \text{d}\omega = \frac{L^3}{2\,\pi^2}\omega^2\left(\frac{1}{\text{v}_l^3}+\frac{2}{\text{v}_\text{tr}^3}\right)\text{d}\omega
\end{equation}
Dabei ist $v_l$ für Longitudinalwellen und $v_\text{tr}$ für Transversalwellen. Ein endlicher Kristall besitzt aber nur $3\,\text{N}_\text{L}$ Eigenschwingungen ($\text{N}_\text{L}$
ist die Anzahl der Atome im Festkörper), sodass eine maximale Frequenz existiert, welche als Debye-Frequenz bezeichnet wird. Aus
\begin{equation}
  \int_0^{\omega_\text{D}} \text{Z} \text{d}\omega = 3\,\text{N}_\text{L}
\end{equation}
folgt
\begin{equation}
  \text{Z}(\omega) = \frac{9\,\text{N}_\text{L}}{\omega_\text{D}^3}\omega^2 \text{d}\omega
\end{equation}
mit Hilfe des Terms für die Debye Frequenz:
\begin{equation}
  \omega_\text{D}^3 = \frac{18\pi^2\text{N}_\text{L}}{\text{L}^3}\frac{1}{\left(\frac{1}{\text{v}-l^3}+\frac{2}{\text{v}_\text{tr}^3}\right)}
  \label{eqn:debye-frequenz}
\end{equation}}
Wird nun die Molwärme berechnet, so definiert sich die Debye-Temperatur ($\theta_\text{D}$) durch
\begin{equation}
  \frac{\theta_\text{D}}{T} = \frac{\hbar\omega_\text{D}}{\text{k}T}.
    \label{eqn:debye-temperatur}
\end{equation}
Damit und mit
\begin{equation}
  x = \frac{\hbar\omega}{\text{k}T}
\end{equation}
ergibt sich eine universelle Funktion, die nicht mehr vom untersuchten Festkörper abhängt:
\begin{equation}
  \text{C}_{\text{V}_\text{Debye}} = 9\,\text{R}\,\left(\frac{T}{\theta_\text{D}}\right)^3 \int_0^{\frac{\theta_\text{D}}{T}} \frac{x^4\text{e}^x}{\left(\text{e}^x - 1\right)^2}\text{d}x = f\left(\frac{\theta_\text{D}}{T}\right)
\end{equation}
Auch diese Funktion zeigt für hohe Temperaturen (T >> $\theta_\text{D}$) ein asymptotisches Verhalten gegen 3$R$. Bei tiefen Temperaturen hingegen (T << $\theta_\text{D}$) wird ein
$T^3$-Verlauf sichtbar, dies wird auch $T^3$-Gesetz genannt.
