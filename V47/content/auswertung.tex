\newpage
\section{Auswertung}
\label{sec:Auswertung}

Die Auswertung erfolgte mit Hilfe von scipy \cite{scipy},
die Fehlerrechnung wurde mit Hilfe von uncertainties \cite{uncertainties}
und die Grafiken wurden mittels matplotlib \cite{matplotlib} erstellt.

Die aufgenommenen Messwerte sind in Tabelle \ref{tab:messwerte}
dargestellt. Es wurde die Heizzeit $\text{d}t$, Heizstrom $I$ und -spannung $U$ der Probe und
die Widerstände der Pt-100 Elemente von Probe und Zylinder notiert. Als Messunsicherheit
wurde für die Heizzeit \SIrange{2}{4}{\second},
für den Heizstrom \SI{0.3}{\milli\ampere},
für die Heizspannung \SI{0.01}{\volt} und
für die Widerstände \SI{0.1}{\ohm} verwendet.

% TODO
% Tabelle mit allen Messwerten
% tab:messwerte

Aus diesen Messwerten wurde $C_\text{p}$ bestimmt nach
\begin{equation}
    C_\text{p} = \frac{U\;\cdot\;I\;\cdot\;\text{d}t}{|T_\text{Probe,f}-T_\text{Probe,i}|\;\cdot\;n}
    \text{,}
\end{equation}
wobei $T_\text{Probe,i}$ die Temperatur der Probe vor dem Heizintervall, $T_\text{Probe,f}$
die Temperatur der Probe nach dem Heizintervall und $n$ die Stoffmenge der Probe beschreibt.
Die Stoffmenge berechnet sich dabei aus dem Quotienten der Masse der Probe
$m_\text{Probe} = \SI{324}{\gram}$ \cite{anleitung} und der Molaren Masse der Probe
$M_\text{Kupfer} = \SI{63.55}{\kilo\gram\per\mole}$ \cite{lenntech} zu
$n \approx \SI{5.4}{\mole}$. Die berechneten $C_\text{p}$-Werte sind in Tabelle
\ref{tab:rechnungen} dargestellt.

Die Umrechnung von $C_\text{p}$ nach $C_\text{V}$ erfolgte mittels der Korrekturformel
\begin{equation}
    C_\text{V} - C_\text{p} = 9\;\alpha^2\;\kappa\;V_0\;T\text{,}
\end{equation}
wobei $\alpha$ der lineare Ausdehnungskoeffizient,
$\kappa = \SI{139e9}{\newton\per\meter\squared}$ \cite{demtroeder} das Kompressionsmodul
und $V_0 = \SI{7.11e-6}{\meter\raiseto{3}\per\mole}$ \cite{webelements} das Molvolumen
von Kupfer ist. Als Temperatur wurde dabei der Mittelwert zwischen der Probentemperatur
zu Beginn und zum Ende der jeweiligen Heizzeit $T_\text{mittel}$ verwendet.
Die sich ergebenden Werte von $T_\text{mittel}$ und $C_\text{V}$ sind in Tabelle
\ref{tab:cv} angegeben und der Verlauf von $C_\text{V}$ ist in Abbildung
\ref{fig:cv} dargestellt.

\begin{figure}
  \centering
  \includegraphics[height=8cm]{build/cv.pdf}      % width=\textwidth
    \caption{Die experimentell ermittelten Werte von $C_\text{V}$ in Abhängigkeit
             von der Temperatur.}
  \label{fig:cv}
\end{figure}
\FloatBarrier

Für den linearen Ausdehnungskoeffizienten $\alpha$ sind in der Versuchsanleitung
\cite[p.~5]{anleitung}
Werte für Kupfer in \SI{10}{\degree}-Schritten angegeben. Diese wurden mittels eines Polynoms
4. Grades der Form
\begin{equation}
    \alpha(T) = a + b\;T + c\;T^2 + d\;T^3 + e\;T^4
\end{equation}
von scipy.curve\_fit gefittet. Dabei ergab sich für die Parameter
\begin{align}
    % TODO
    .
\end{align}
Der Verlauf von $\alpha$ ist in Abbildung \ref{fig:alpha} dargestellt.
Die für die Berechnung von $C_\text{V}$ verwendeten Werte von $\alpha$ sind ebenfalls
in Tabelle \ref{tab:cv} aufgeführt.

\begin{figure}
  \centering
  \includegraphics[height=8cm]{build/alpha.pdf}      % width=\textwidth
    \caption{Messwerte und Regression des linearen Ausdehnungskoeffizienten.}
  \label{fig:alpha}
\end{figure}

\FloatBarrier
Aus den gemessenen $(C_\text{V}, T)$-Wertepaaren wurde bis zu einer Temperatur von
\SI{170}{\kelvin} mittels einer Tabelle in der
Versuchsanleitung ein Wert für $\sfrac{\theta_\text{D}}{T}$ ermittelt. Diese wurden
mit der mittleren Temperatur $T_\text{mittel}$ multipliziert und sind ebenfalls
in Tabelle \ref{tab:cv} aufgeführt. Der Mittelwert von $\theta_\text{D}$ ergab sich
zu % TODO.


% TODO
% Tabelle mit Cp-Werten, Tmittel-Werten, alpha-Werten, Cv-Werten, thetaD/T, thetaD
% tab:cv


