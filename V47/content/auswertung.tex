\newpage
\section{Auswertung}
\label{sec:Auswertung}

Die Auswertung erfolgte mit Hilfe von scipy \cite{scipy},
die Fehlerrechnung wurde mit Hilfe von uncertainties \cite{uncertainties}
und die Grafiken wurden mittels matplotlib \cite{matplotlib} erstellt.

Die aufgenommenen Messwerte sind in Tabelle \ref{tab:messwerte}
dargestellt. Es wurde die Heizzeit $\text{d}t$, Heizstrom $I$ und -spannung $U$ der Probe und
die Widerstände der Pt-100 Elemente von Probe und Zylinder notiert. Als Messunsicherheit
wurde für die Heizzeit \SIrange{2}{4}{\second},
für den Heizstrom \SI{0.3}{\milli\ampere},
für die Heizspannung \SI{0.01}{\volt} und
für die Widerstände \SI{0.1}{\ohm} verwendet.
Die Widerstände der Pt-100 Elemente wurden mittels der Relation
\begin{equation}
    T(R) = \num{0.00134}\;R^2 + \num{2.296}\;R - \num{243.02}
\end{equation}
in Temperaturen von Probe und Zylinder umgerechnet.

% Tabelle mit allen Messwerten
% tab:messwerte
\begin{table}
    \centering
    \caption{Aufgenommene und umgerechnete Messwerte.}
    \label{tab:messwerte}
    \sisetup{table-format=3.1}
    \begin{tabular}{S[table-format=3.0]
                    S[table-format=2.2]
                    S
                    S
                    S
                    S @{${}\pm{}$} S[table-format=1.1]
                    S @{${}\pm{}$} S[table-format=1.1]}
    \toprule
        {$\mathrm{d}t \: / \: \si{\second}$} &
        {$U \: / \: \si{\volt}$} &
        {$I \: / \: \si{\milli\ampere}$} &
        {$R_\text{Probe,i} \: / \: \si{\ohm}$} &
        {$R_\text{Probe,f} \: / \: \si{\ohm}$} &
        \multicolumn{2}{c}{$T_\text{Probe,i} \: / \: \si{\degree}$} &
        \multicolumn{2}{c}{$T_\text{Probe,f} \: / \: \si{\degree}$} \\
    \midrule
    300 & 14.19 & 192.5 & 22.2 & 28.1 & -191.4 & 0.2 & -181.7 & 0.2 \\
    210 & 14.20 & 194.5 & 28.1 & 31.9 & -177.4 & 0.2 & -172.9 & 0.2 \\
    210 & 14.20 & 195.2 & 31.9 & 35.5 & -168.4 & 0.2 & -154.3 & 0.2 \\
    240 & 14.20 & 192.2 & 35.7 & 40.2 & -159.3 & 0.2 & -117.6 & 0.2 \\
    240 & 14.21 & 192.0 & 40.3 & 44.8 & -148.3 & 0.2 & -93.1 & 0.2 \\
    240 & 14.22 & 189.5 & 44.8 & 48.9 & -137.5 & 0.2 & -89.8 & 0.2 \\
    240 & 14.22 & 190.0 & 48.9 & 52.9 & -127.5 & 0.2 & -91.6 & 0.2 \\
    240 & 14.23 & 190.0 & 52.9 & 56.6 & -117.8 & 0.2 & -93.1 & 0.2 \\
    240 & 14.24 & 190.0 & 56.6 & 60.0 & -108.8 & 0.2 & -90.6 & 0.2 \\
    240 & 14.24 & 190.0 & 60.0 & 63.3 & -100.4 & 0.2 & -79.7 & 0.2 \\
    270 & 14.25 & 190.1 & 63.3 & 67.1 & -92.3 & 0.2 & -74.2 & 0.2 \\
    270 & 14.25 & 190.2 & 67.1 & 70.7 & -82.9 & 0.2 & -68.5 & 0.2 \\
    270 & 14.26 & 190.3 & 70.7 & 74.2 & -74.0 & 0.2 & -62.0 & 0.2 \\
    270 & 14.26 & 190.3 & 74.2 & 77.6 & -65.3 & 0.2 & -46.5 & 0.3 \\
    270 & 14.26 & 190.4 & 77.6 & 81.1 & -56.8 & 0.3 & -37.4 & 0.3 \\
    270 & 14.27 & 190.5 & 81.1 & 85.0 & -48.0 & 0.3 & -32.9 & 0.3 \\
    270 & 14.27 & 190.9 & 85.0 & 88.4 & -38.2 & 0.3 & -25.8 & 0.3 \\
    270 & 14.27 & 190.8 & 88.6 & 92.2 & -29.1 & 0.3 & -12.3 & 0.3 \\
    270 & 14.28 & 191.1 & 92.2 & 96.0 & -19.9 & 0.3 & -3.4 & 0.3 \\
    270 & 14.27 & 189.7 & 96.2 & 99.7 & -9.7 & 0.3 & 2.0 & 0.3 \\
    270 & 14.27 & 189.9 & 99.7 & 103.0 & -0.8 & 0.3 & 11.3 & 0.3 \\
    270 & 14.28 & 191.7 & 103.0 & 106.4 & 7.7 & 0.3 & 20.8 & 0.3 \\
    270 & 14.28 & 191.5 & 106.4 & 109.8 & 16.4 & 0.3 & 29.9 & 0.3 \\
    270 & 14.28 & 191.6 & 109.8 & 113.2 & 25.2 & 0.3 & 39.0 & 0.3 \\
    270 & 14.28 & 191.7 & 113.2 & 116.7 & 34.1 & 0.3 & 43.4 & 0.3 \\
\end{tabular}
\end{table}


Aus diesen Messwerten wurde $C_\text{p}$ bestimmt nach
\begin{equation*}
    C_\text{p} = \frac{U\;\cdot\;I\;\cdot\;\text{d}t}{|T_\text{Probe,f}-T_\text{Probe,i}|\;\cdot\;n}
    \text{,}
\end{equation*}
wobei $T_\text{Probe,i}$ die Temperatur der Probe vor dem Heizintervall, $T_\text{Probe,f}$
die Temperatur der Probe nach dem Heizintervall und $n$ die Stoffmenge der Probe beschreibt.
Die Stoffmenge berechnet sich dabei aus dem Quotienten der Masse der Probe
$m_\text{Probe} = \SI{324}{\gram}$ \cite{anleitung} und der molaren Masse der Probe
$M_\text{Kupfer} = \SI{63.55}{\gram\per\mole}$ \cite{lenntech} zu
$n \approx \SI{5.4}{\mole}$. Die berechneten $C_\text{p}$-Werte sind in Tabelle
\ref{tab:cv} dargestellt.

Die Umrechnung von $C_\text{p}$ nach $C_\text{V}$ erfolgte mittels der Korrekturformel
\begin{equation}
    C_\text{V} - C_\text{p} = 9\;\alpha^2\;\kappa\;V_0\;T\text{,}
    \label{eqn:TausR}
\end{equation}
wobei $\alpha$ der lineare Ausdehnungskoeffizient,
$\kappa = \SI{139e9}{\newton\per\meter\squared}$ \cite{demtroeder} das Kompressionsmodul
und $V_0 = \SI{7.11e-6}{\raiseto{3}\meter\per\mole}$ \cite{webelements} das Molvolumen
von Kupfer ist. Als Temperatur wurde dabei der Mittelwert zwischen der Probentemperatur
zu Beginn und zum Ende der jeweiligen Heizzeit $T_\text{mittel}$ verwendet.
Die sich ergebenden Werte von $T_\text{mittel}$ und $C_\text{V}$ sind in Tabelle
\ref{tab:cv} angegeben und der Verlauf von $C_\text{V}$ ist in Abbildung
\ref{fig:cv} dargestellt.

\begin{figure}
  \centering
  \includegraphics[height=8cm]{build/cv.pdf}      % width=\textwidth
    \caption{Die experimentell ermittelten Werte von $C_\text{V}$ in Abhängigkeit
             von der Temperatur.}
  \label{fig:cv}
\end{figure}
\FloatBarrier

% Tabelle mit Cp-Werten, Tmittel-Werten, alpha-Werten, Cv-Werten
% tab:cv
\begin{table}
    \centering
    \caption{Zwischenergebnisse zur Berechnung von $C_\text{V}.$}
    \label{tab:cv}
    \sisetup{table-format=1.2}
    \begin{tabular}{S[table-format=2.2] @{${}\pm{}$} S
                    S[table-format=3.1] @{${}\pm{}$} S[table-format=1.1]
                    S[table-format=2.2] @{${}\pm{}$} S
                    S[table-format=2.2] @{${}\pm{}$} S}
    \toprule
        \multicolumn{2}{c}{$C_\text{p} \: / \: \si{\joule\raiseto{-1}\mole\raiseto{-1}\degree}$} &
        \multicolumn{2}{c}{$T_\text{mittel} \: / \: \si{\degree}$} &
        \multicolumn{2}{c}{$\alpha \: / \: 10^6\si{\raiseto{-1}\degree}$} &
        \multicolumn{2}{c}{$C_\text{V} \: / \: \si{\joule\raiseto{-1}\mole\raiseto{-1}\degree}$} \\
    \midrule
    15.73 & 0.57 & -186.5 & 0.2 & 9.27 & 2.08 & 15.87 & 0.57 \\
    23.90 & 1.81 & -175.2 & 0.2 & 10.45 & 1.73 & 24.07 & 1.81 \\
    7.67 & 0.21 & -161.4 & 0.2 & 11.59 & 1.36 & 7.87 & 0.22 \\
    2.91 & 0.04 & -138.5 & 0.2 & 12.96 & 0.89 & 3.12 & 0.05 \\
    2.20 & 0.03 & -120.7 & 0.2 & 13.69 & 0.62 & 2.40 & 0.04 \\
    2.52 & 0.04 & -113.7 & 0.2 & 13.93 & 0.53 & 2.72 & 0.04 \\
    3.35 & 0.05 & -109.6 & 0.2 & 14.06 & 0.48 & 3.54 & 0.05 \\
    4.87 & 0.09 & -105.4 & 0.2 & 14.18 & 0.44 & 5.06 & 0.09 \\
    6.63 & 0.15 & -99.7 & 0.2 & 14.33 & 0.38 & 6.82 & 0.15 \\
    5.82 & 0.12 & -90.1 & 0.2 & 14.57 & 0.3 & 5.99 & 0.12 \\
    7.52 & 0.17 & -83.3 & 0.2 & 14.73 & 0.25 & 7.68 & 0.17 \\
    9.44 & 0.25 & -75.7 & 0.2 & 14.90 & 0.2 & 9.59 & 0.25 \\
    11.38 & 0.36 & -68.0 & 0.2 & 15.06 & 0.16 & 11.52 & 0.36 \\
    7.25 & 0.16 & -55.9 & 0.2 & 15.30 & 0.11 & 7.36 & 0.16 \\
    7.04 & 0.15 & -47.1 & 0.2 & 15.48 & 0.08 & 7.14 & 0.15 \\
    9.02 & 0.24 & -40.4 & 0.2 & 15.61 & 0.06 & 9.10 & 0.24 \\
    11.02 & 0.34 & -32.0 & 0.2 & 15.77 & 0.05 & 11.09 & 0.34 \\
    8.14 & 0.20 & -20.7 & 0.2 & 15.98 & 0.03 & 8.19 & 0.20 \\
    8.25 & 0.20 & -11.6 & 0.2 & 16.14 & 0.03 & 8.28 & 0.20 \\
    11.53 & 0.38 & -3.9 & 0.2 & 16.26 & 0.02 & 11.54 & 0.38 \\
    11.26 & 0.36 & 5.2 & 0.2 & 16.39 & 0.02 & 11.25 & 0.36 \\
    10.44 & 0.31 & 14.3 & 0.2 & 16.50 & 0.03 & 10.41 & 0.31 \\
    10.19 & 0.30 & 23.2 & 0.2 & 16.57 & 0.03 & 10.14 & 0.30 \\
    9.97 & 0.29 & 32.1 & 0.2 & 16.60 & 0.05 & 9.89 & 0.29 \\
    14.65 & 0.60 & 38.7 & 0.2 & 16.59 & 0.06 & 14.55 & 0.60 \\
    \end{tabular}
\end{table}

\FloatBarrier

Für den linearen Ausdehnungskoeffizienten $\alpha$ sind in der Versuchsanleitung
\cite[p.~5]{anleitung}
Werte für Kupfer in \SI{10}{\degree}-Schritten angegeben. Diese wurden mittels eines Polynoms
4. Grades der Form
\begin{equation*}
    \alpha(T) = a + b\;T + c\;T^2 + d\;T^3 + e\;T^4
\end{equation*}
von scipy.curve\_fit gefittet. Dabei ergab sich für die Parameter
\begin{align*}
    a &= \SI{1.632(2)e-05}{1\per\grd} \\
    b &= \SI{1.45(9)e-08}{1\per\raiseto{2}\grd} \\
    c &= \SI{-1.3(3)e-10}{1\per\raiseto{3}\grd} \\
    d &= \SI{-1.5(3)e-12}{1\per\raiseto{4}\grd} \\
    e &= \SI{-8.2(7)e-15}{1\per\raiseto{5}\grd}
\end{align*}
Der Verlauf von $\alpha$ ist in Abbildung \ref{fig:alpha} dargestellt.
Die für die Berechnung von $C_\text{V}$ verwendeten Werte von $\alpha$ sind ebenfalls
in Tabelle \ref{tab:cv} aufgeführt.

\begin{figure}
  \centering
  \includegraphics[height=8cm]{build/alpha.pdf}      % width=\textwidth
    \caption{Messwerte und Regression des linearen Ausdehnungskoeffizienten.}
  \label{fig:alpha}
\end{figure}
\FloatBarrier

\FloatBarrier
Aus den gemessenen $(C_\text{V}, T)$-Wertepaaren wurde bis zu einer Temperatur von
\SI{170}{\kelvin} mittels einer Tabelle in der
Versuchsanleitung \cite[p.~5]{anleitung} ein Wert für $\sfrac{\theta_\text{D}}{T}$ ermittelt. Diese wurden
mit der mittleren Temperatur $T_\text{mittel}$ multipliziert und sind
in Tabelle \ref{tab:theta} aufgeführt. Der Mittelwert von $\theta_\text{D}$ ergab sich
zu \SI{915.9(4)}{\kelvin}.

% Tabelle mit ThetaD/T, thetaD
% tab:theta
\begin{table}
    \centering
    \caption{Verwendete Größen zur Bestimmung der Debye-Temperatur.}
    \label{tab:theta}
    \sisetup{table-format=1.1}
    \begin{tabular}{S[table-format=3.2] @{${}\pm{}$} S
                    S[table-format=2.2] @{${}\pm{}$} S[table-format=1.2]
                    S
                    S[table-format=4.1] @{${}\pm{}$} S}
    \toprule
        \multicolumn{2}{c}{$T_\text{mittel} \: / \: \si{\kelvin}$} &
        \multicolumn{2}{c}{$C_\text{V} \: / \: \si{\joule\raiseto{-1}\mole\raiseto{-1}\grd}$} &
        {$\sfrac{\theta_\text{D}}{T}$} &
        \multicolumn{2}{c}{$\theta_\text{D,exp} \: / \: \si{\kelvin}$} &
    \midrule
    -186.5 & 0.2 & 15.87 & 0.57 & 3.2 & 277.1 & 0.5 \\
    -175.2 & 0.2 & 24.07 & 1.81 & 0.8 & 78.4 & 0.1 \\
    -161.4 & 0.2 & 7.87 & 0.22 & 5.5 & 614.8 & 0.9 \\
    -138.5 & 0.2 & 3.12 & 0.05 & 8.3 & 1118.0 & 1.4 \\
    -120.7 & 0.2 & 2.40 & 0.04 & 9.2 & 1402.7 & 1.6 \\
    -113.7 & 0.2 & 2.72 & 0.04 & 8.8 & 1403.5 & 1.5 \\
    -109.6 & 0.2 & 3.54 & 0.05 & 7.9 & 1292.4 & 1.4 \\
    -105.4 & 0.2 & 5.06 & 0.09 & 6.8 & 1140.5 & 1.2 \\
    \end{tabular}
\end{table}

