\section{Diskussion}
\label{sec:Diskussion}

Die in Tabelle \ref{tab:2,3} aufgeführten Absorptionskoeffizienten der Würfel mit
homogener Materialverteilung werden mit den Literaturwerten in Tabelle \ref{tab:lit} verglichen.
Daraus ergibt sich, dass der Koeffizient des Würfel 2 $\mu = \SI{0.177}{\per\centi\meter}$
gut zu Aluminium passt. Die Abweichung beträgt dabei $\SI{12.38}{\percent}$.
Bei Würfel 3 ($\mu=\SI{0.918}{\per\centi\meter}$) beträgt die Abweichung zu Blei
$\SI{26.56}{\percent}$, weshalb davon auszugehen ist, dass der Würfel 3 aus Blei besteht.

\begin{table}[htb]
  \centering
  \caption{Die Absorptionskoeffizienten verschiedener Materialien \cite{koef}.}
      \begin{tabular}{c
                      S[table-format=1.3]
      								S[table-format=2.2]
      								S[table-format=1.3]}
        \toprule
        {Material} & {$\sigma$, $\si{\centi\meter\squared\per\gram}$} & {$\rho$, $\si{\gram\per\centi\meter^{3}}$} & {$\mu$, $\si{\per\centi\meter}$} \\
      	\midrule
        Blei & 0.110 & 11.34 & 1.245 \\
        Messing & 0.073 & 8.41 & 0.614 \\
        Eisen & 0.073 & 7.86 & 0.574 \\
        Aluminium & 0.075 & 2.71 & 0.203 \\
        Delrin & 0.082 & 1.42 & 0.116 \\
        \bottomrule
      \end{tabular}
  \label{tab:lit}
\end{table}

Aus der Versuchsanleitung und den Messergebnissen geht hervor, dass Würfel 4 aus
verschiedenen Teilwürfeln besteht.
Aufgrund der weiter unten genannten Gründe kommt es zu Abweichungen der
Zählraten nach unten. Deshalb wurde im Zweifelsfall bei der Materialbestimmung dasjenige
Material ausgewählt, welches den nächst höheren Absorptionskoeffizienten
hat.

\begin{table}[htb]
  \centering
  \caption{Zusammensetzung des 4. Würfels anhand der Absorptionskoeffizienten.}
  \begin{tabular}{c
                  S[table-format=1.4(1)]
                  c
                  S}
          \toprule
          {Teilwürfel} & {Absorptionskoeffizienten $\mu$, $\si{\per\centi\meter}$} & {Material} & {Abweichung, $\si{\percent}$} \\
          \midrule
          1 & 1.137 & Blei & 6.56 \\
          2 & 0.608 & Messing & 1.00 \\
          3 & 0.938 & Blei & 24.96 \\
          4 & 0.264 & Aluminium & 30.69 \\
          5 & 1.322 & Blei & 5.76 \\
          6 & 0.873 & Messing & 45.02 \\
          7 & 1.267 & Blei & 1.36 \\
          8 & 0.361 & Aluminium & 78.71 \\
          9 & 0.991 & Blei & 20.72 \\
  \end{tabular}
  \label{tab:ergebnisse}
\end{table}

Bei dieser Zuordnung ist anzumerken, dass es einige experimentelle Schwierigkeiten gab,
die zu großen Fehlern geführt haben können. Zum einen wurde der NaJ-Detektor nicht vor
der Datennahme vorgewärmt. Dies hat zu einer Abweichung in den Zählraten nach oben geführt,
was sich der deutlich kleineren Zählrate der zweiten Nullmessung entnehmen lässt. 
Zu Beginn ergaben sich daher wesentlich höhere Zählraten als bei den späteren Messungen. Damit lässt 
sich ein Teil der Abweichungen in den Ergebnissen erklären. Dazu kommt, dass die Justierung per Hand
nicht genau genug eingestellt werden konnte. Auch das Programm, welches für die Datenaufnahme verwendet wurde
stellt eine Fehlerquelle dar, da dies den Untergrund direkt abzieht. Dies führt zu nicht ausgleichbaren
systematischen Fehlern.
% Die radioaktive Quelle stellt zudem keinen perfekt fokussierten Strahl zur Verfügung.
Da  der Zerfall der Quelle ein rein
statistischer Prozess ist, kann es zudem zu zeitlichen Fluktuationen kommen.
Diese wurden jedoch wie am Anfang der Auswertung beschrieben auf ein
kleines Maß beschränkt.
