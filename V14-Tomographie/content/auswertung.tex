\newpage
\section{Auswertung}

\label{sec:Auswertung}

Als Fehler wird ein statistischer Poissonfehler verwendet, da es sich in diesem
Experiment um ein Zählexperiment handelt. Dieser soll hier unter \SI{3}{\percent} liegen,
damit auf genauere Rechnungen verzichtet werden kann.

\begin{equation*}
  \frac{1}{\sqrt{N}} < 0.03
\end{equation*}

Ab wann dies gewährleistet ist, ist in der Durchführung \ref{sec:Aufbau} erläutert.

\subsection{Nullmessung}
Zur Bestimmung der Spektren der Würfel wird zunächst eine Nullmessung (Messung
ohne Würfel im Strahlengang) durchgeführt. Diese ist in Abbildung \ref{fig:leer}
dargestellt. Gemessen wird eine Rate von
\begin{equation*}
    I_0 = 158.6 \,\frac{\text{Zählungen}}{\text{s}}
\end{equation*}
bei $9520$ Ereignissen über ein Zeitintervall von $\SI{60}{\second}$ verwendet.

\begin{figure}[htb]
  \centering
  \includegraphics[width=0.75\textwidth]{build/Nullmessung1.pdf}
  \caption{Messung der Energieverteilung bei einer Leermessung ohne Würfel. Die
  Fehlerbalken repräsentieren die statistischen Poissonfehler.}
  \label{fig:leer}
\end{figure}
\FloatBarrier
Da die einzelnen Würfel ein Aluminiumgehäuse besitzen, wird eine Messung eines solchen
Gehäuses durchgeführt, um die zuätzliche Absorption zu ermitteln. Dieses Aluminiumgehäuse
sorgt dafür, dass die einzelnen kleinen Würfelteile nicht auseinanderfallen. Das
gemessene Spektrum des Aluminiumgehäuses ist in
Abbildung \ref{fig:alu} zu sehen.

\begin{figure}[htb]
  \centering
  \includegraphics[width=0.75\textwidth]{build/W1_pos1.pdf}
  \caption{Messung der Energieverteilung eines Aluminiumgehäuses. Die Fehlerbalken
  repräsentieren die statistischen Poissonfehler.}
  \label{fig:alu}
\end{figure}
\FloatBarrier
\newside
Da ein Würfel durch Drehen und Verschieben einen unterschiedlichen Wegunterschied
besitzen kann, werden die in Abbildung \ref{fig:wuerfelpos} zu erkennenden Projektionen
$I_1$, $I_2$, $I_{11}$ und $I_{12}$ betrachtet. Die Intensitätsvektoren $\vec{I}$ des Würfels ergeben sich aus
den gemessenen Daten. $\vec{\tilde{I}}$ wird dabei durch Formel \eqref{eqn:matrixform} berechnet.
Dabei ist zu sehen, das allerdings mehrere Projektionen die selben Weglängen
besitzen, füllt man die zueinender analogen Projektionen mit den entsprechenden Werten,
so ergibt sich der Vektor $\vec{\tilde{I}}$:

\begin{equation*}
  \vec{I}=\begin{pmatrix}
        - \\
        - \\
        - \\
        156.68  \\
        157.28  \\
        - \\
        - \\
        - \\
        - \\
        156.82 \\
        156.53 \\
        -
  \end{pmatrix},\
  \vec{\tilde{I}}=\begin{pmatrix}
        156.68 \\
        157.28 \\
        156.68 \\
        156.68 \\
        157.28 \\
        156.68 \\
        156.82 \\
        156.53 \\
        156.82 \\
        156.82 \\
        156.53 \\
        156.82
  \end{pmatrix}
\end{equation*}


\subsection{Würfel aus einheitlichem Material}
Die verwendeten Würfel 2 und 3 bestehen nach der Anleitung \cite{anleitung} einheitlich aus Aluminum bzw.
Blei. Zur Betrachtung werden die Messungen des Gehäuses von den Messdaten der anderen Würfel abgezogen,
um nur den Inhalt zu Betrachten. Die Geometriematrix kann nun in einen
Vektor mit aufsummierten Komponenten umgeschrieben werden. Dabei ergibt sich für beide
Würfel der selbe Vektor, da beide Würfel gleich durchstahlt werden:

\begin{equation*}
  A_2 = A_3 = \begin{pmatrix}
          3 \\
          3 \\
          3\sqrt{2} \\
          3\sqrt{2}
        \end{pmatrix}
\end{equation*}
Diese Vektoren beschreiben die Projektionen $I_1$, $I_2$, $I_{11}$ und $I_{12}$.
Aus der Methode der kleinsten Quadrate (Formeln \eqref{eqn:least-squares} und \eqref{eqn:least-squares-error})
lassen sich die einzelnen Absorptionskoeffizienten der verschiedenen Materialien berechnen.
Die daraus bestimmten Werte sind in Tabelle \ref{tab:2,3} nachzulesen.


 \begin{table}[htb]
   \centering
   \caption{Absorptionskoeffizienten der Würfel 2 und 3 bei verschiedenen Projektionen.}
   \begin{tabular}{c
                S[table-format=1.3(1)]
                c
                c
                c}
        \toprule
        {Würfel} & {Absorptionkoeffizient $\mu$, $\si{\per\centi\meter}$} & {Sollmaterial} & $\mu$ Literatur & Abweichung/\%\\
        \midrule
        2 & 0.18(2) & Aluminium & 0.203 & 11,33\\
        3 & 0.91(6) & Blei & 1.245 & 26,90\\
   \end{tabular}
   \label{tab:2,3}
 \end{table}
 Die Herkunft der Abweichungen wird in Abschnitt \ref{sec:Diskussion} beschrieben.

 \subsection{Würfel mit unbekanntem Inhalt}
 Vor der Untersuchung des Würfel 4 wird die Intensität ohne Würfel im Strahlengang noch
 einmal gemessen. Dabei wirdn 6916 Ereignisse in 60$\,$s gezählt, also eine Rate von

 \begin{equation*}
   I_0 = 115.26 \,\frac{\text{Zählungen}}{\si{s}}.
 \end{equation*}
Dies ergibt einen deutlich anderen Wert als die erste Nullmessung.

 Der Würfel 4 wird untersucht, um sein Material zu bestimmen. Dafür wird er mit 12
 Projektionen vermessen. Die neun einzelnen
 Absorptionskoeffizienten werden durch die Geometriematrix $A$ bestimmt. Diese ergibt
 sich durch den Weg, den die Teilchen durch den Würfel zurücklegen, um zum Detektor zu
 gelangen. Mit den einzelnen Projektionen, die in \ref{fig:wuerfelpos} zu erkennen
 sind, werden die einzelnen Zeilen gefüllt. Dabei steht eine $\sqrt{2}$ für eine
 Projektion durch eine Diagonale. Daraus ergibt sich $A$ mit der Form

 \begin{equation*}
   A = \begin{pmatrix}
              1 & 0 & 0 & 1 & 0 & 0 & 1 & 0 & 0 \\
              0 & 1 & 0 & 0 & 1 & 0 & 0 & 1 & 0 \\
              0 & 0 & 1 & 0 & 0 & 1 & 0 & 0 & 1 \\
              1 & 1 & 1 & 0 & 0 & 0 & 0 & 0 & 0 \\
              0 & 0 & 0 & 1 & 1 & 1 & 0 & 0 & 0 \\
              0 & 0 & 0 & 0 & 0 & 0 & 1 & 1 & 1 \\
              0 & \sqrt{2} & 0 & \sqrt{2} & 0 & 0 & 0 & 0 & 0 \\
              0 & 0 & \sqrt{2} & 0 & \sqrt{2} & 0 & \sqrt{2} & 0 & 0 \\
              0 & 0 & 0 & 0 & 0 & \sqrt{2} & 0 & \sqrt{2} & 0 \\
              0 & 0 & 0 & \sqrt{2} & 0 & 0 & 0 & \sqrt{2} & 0 \\
              \sqrt{2} & 0 & 0 & 0 & \sqrt{2} & 0 & 0 & 0 & \sqrt{2} \\
              0 & \sqrt{2} & 0 & 0 & 0 & \sqrt{2} & 0 & 0 & 0
      \end{pmatrix}.
 \end{equation*}

Der dazugehörige Vektor der Ausgangszählraten $I_4$ und der Vektor der
logarithmierten Intensitäten der
Projektionen bestehen aus:

\begin{equation*}
  \vec{I_4} = \begin{pmatrix}
      6.65 \pm 0.12 \\
      4.83 \pm 0.10  \\
      4.86 \pm 0.10  \\
      8.81 \pm 0.14  \\
      7.86 \pm 0.13  \\
      9.52 \pm 0.14  \\
      7.99 \pm 0.13  \\
      3.21 \pm 0.08  \\
      5.52 \pm 0.11  \\
      13.22 \pm 0.17  \\
      3.38 \pm 0.8 \\
      5.36 \pm 0.11
  \end{pmatrix},\
   \vec{\tilde{I_4}} = \begin{pmatrix}
   3.15 \pm 0.38 \\
   3.48 \pm 0.46 \\
   3.47 \pm 0.45 \\
   2.88 \pm 0.33 \\
   2.99 \pm 0.36 \\
   2.80 \pm 0.32 \\
   -1.60 \pm 1.06 \\
   3.89 \pm 0.56 \\
   -1.22 \pm 1.09 \\
   -2.10 \pm 1.04 \\
   3.84 \pm 0.54 \\
   -1.1 \pm 1.10
 \end{pmatrix}
\end{equation*}

Analog zu den Würfeln 2 und 3 wird auch hier die Methode der kleinsten Quadrate
angewandt. Mit den Ergebnissen und dem normalisierten Logarithmus ergeben sich
folgende Werte für die Absorptionskoeffizienten:

\begin{table}
  \centering
  \caption{Bestimmte Absorptioinskoeffizienten aus den verschiedenen Projektionen für Würfel 4.}
  \label{tab:5}
  \begin{tabular}{c
                  S[table-format=1.4(1)]}
                  \toprule
                  {Teilwürfel} & {Absorptionskoeffizient $\mu$, $\si{\per\centi\meter}$}\\
                  \midrule
                  1 & 1.137(363)\\
                  2 & 0.608(414)\\
                  3 & 0.938(369)\\
                  4 & 0.264(407)\\
                  5 & 1.322(234)\\
                  6 & 0.872(412)\\
                  7 & 1.267(361)\\
                  8 & 0.360(412)\\
                  9 & 0.991(369)\\
  \end{tabular}
\end{table}
