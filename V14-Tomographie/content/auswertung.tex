\newpage
\section{Auswertung}

\label{sec:Auswertung}
Als Fehler wird ein statistischer Poissonfehler verwendet, da es sich in diesem
Experiment um ein Zählexperiment handelt. Dieser soll hier unter 3\% liegen,
damit auf genauere Rechnungen verzichtet werden kann:

\begin{equation*}
  \frac{1}{\sqrt{N}} < 0.03
\end{equation*}

Damit dies gewährleistet ist, wird bei jeder Messung mindestens eine
Zählrate von $\text{N}>1112$ aufgenommen.
Zur Bestimmung der Spektren der Würfel wird zunächst eine Nullmessung (Messung
ohne Würfel im Strahlengang) durchgeführt. Durch dies wird Abbildung \ref{fig:leer}
erhalten. Somit ergibt sich eine Rate von
\begin{equation*}
    I_0 = 158.6 \,\frac{\text{Zählungen}}{\text{s}}
\end{equation*}
bei $9520$ Ereignissen über ein Zeitintervall von $\SI{60}{\second}$.

\begin{figure}[htb]
  \centering
  \includegraphics[width=0.8\textwidth]{build/Nullmessung1.pdf}
  \caption{Messung der Energieverteilung bei einer Leermessung ohne Würfel. Die
  Fehlerbalken repräsentieren die statistischen Poissonfehler.}
  \label{fig:leer}
\end{figure}

Da die einzelnen Würfel ein Aluminiumgehäuse besitzen wird eine Mesung eines solchen
Gehäuses durchgeführt um die zuätzliche Absorption zu ermitteln. Das gemessene Spektrum ist in
Abbildung \ref{fig:alu} zu sehen.

\begin{figure}[htb]
  \centering
  \includegraphics[width=0.8\textwidth]{build/W1_pos1.pdf}
  \caption{Messung der Energieverteilung eines Aluminiumgehäuses. Die Fehlerbalken
  repräsentieren die statistischen Poissonfehler.}
  \label{fig:alu}
\end{figure}

Da ein Würfel durch Drehen und Verschieben einen unterschiedlichen Wegunterschied
besitzen kann, werden die in Abbildung \ref{fig:strahlengänge} zu erkennenden Projektionen
------ betrachtet. Die Intensitätsvektoren $\vec{I}$ des Würfels ergeben sich aus
den gemessenen Daten. Dabei ist zu sehen, das allerdings mehrere Projektionen die selben Weglängen
besitzen, daraus ergibt sich der Vektor $\vec{\tilde{I}}$:

\begin{equation*}
  \vec{I}=\begin{pmatrix}
        - \\
        - \\
        - \\
        156.68  \\
        157.28  \\
        - \\
        - \\
        - \\
        - \\
        156.82 \\
        156.53 \\
        -
  \end{pmatrix},\
  \vec{\tilde{I}}=\begin{pmatrix}
        156.68 \\
        157.28 \\
        156.68 \\
        156.68 \\
        157.28 \\
        156.68 \\
        156.82 \\
        156.53 \\
        156.82 \\
        156.82 \\
        156.53 \\
        156.82
  \end{pmatrix}
\end{equation*}


\subsection{Würfel aus einheitlichem Material}
Die verwendeten Würfel 2 und 3 bestehen nach der Anleitung einheitlich aus Aluminum bzw.
Blei und die Projektionen können gemittelt werden. Die Geometriematrix kann nun in einen
Vektor mit aufsummierten Komponenten umgeschrieben werden. Dabei ergibt sich für beide
Würfel der selbe Vektor, da beide Würfel gleich durchstahlt wurden:

\begin{equation*}
  A_2 = A_3 = \begin{pmatrix}
          3 \\
          3 \\
          3\sqrt{2} \\
          3\sqrt{2}
        \end{pmatrix}
\end{equation*}

Diese Vektoren beschreiben die Projektionen % TODO #####.
Aus der Methode der kleinsten Quadrate
lassen sich die einzelnen Absorptionskoeffizienten der verschiedenen Materialien berechnen.
Die bestimmten Werte der Absoptionskoeffizienten sind in Tabelle \ref{tab:2,3} nachzulesen.

 \begin{table}[htb]
   \centering
   \caption{Absorptionskoeffizienten der Würfel 2 und 3 bei verschiedenen Projektionen.}
   \begin{tabular}{c
                S[table-format=1.3(1)]
                c}
        \toprule
        {Würfel} & {Absorptionkoeffizient $\mu$, $\si{\per\centi\meter}$} & {Sollmaterial}\\
        \midrule
        2 & 0.18(2) & Aluminium \\
        3 & 0.91(6) & Blei \\
   \end{tabular}
   \label{tab:2,3}
 \end{table}

 \subsection{Würfel mit unbekanntem Inhalt}
 Vor der Untersuchung des Würfel 4 wurde die Intensität ohne Würfel im Strahlengang noch
 einmal gemessen. Dabei wurde, 6916 Ereignisse in 60$\,$s gezählt, also eine Rate von

 \begin{equation*}
   I_0 = 115.26 \,\frac{\text{Zählungen}}{\text{s}}.
 \end{equation*}

 Das Spektrum dieser Messung ist in Abb. \ref{fig:leer2} zu finden.

 \begin{figure}[htb]
   \centering
   \includegraphics[width=0.8\textwidth]{build/Nullmessung2.pdf}
   \caption{Messung des Energiespektrums ohne Würfel im Strahlengang vor der Vermessung
   von Würfel 4.}
   \label{fig:leer2}
 \end{figure}

 Der Würfel 4 wird untersucht, da sein Inhalt bestimmt werden soll. Dafür wird er durch 12
 Projektionen vermessen. Da die Substruktur des Würfels unbekannt ist werden die neun einzelnen
 Absorptionskoeffizienten durch die Geometriematrix $A$ bestimmt. $A$ hat die Form:

 \begin{equation*}
   A = \begin{pmatrix}
              1 & 0 & 0 & 1 & 0 & 0 & 1 & 0 & 0 \\
              0 & 1 & 0 & 0 & 1 & 0 & 0 & 1 & 0 \\
              0 & 0 & 1 & 0 & 0 & 1 & 0 & 0 & 1 \\
              1 & 1 & 1 & 0 & 0 & 0 & 0 & 0 & 0 \\
              0 & 0 & 0 & 1 & 1 & 1 & 0 & 0 & 0 \\
              0 & 0 & 0 & 0 & 0 & 0 & 1 & 1 & 1 \\
              0 & \sqrt{2} & 0 & \sqrt{2} & 0 & 0 & 0 & 0 & 0 \\
              0 & 0 & \sqrt{2} & 0 & \sqrt{2} & 0 & \sqrt{2} & 0 & 0 \\
              0 & 0 & 0 & 0 & 0 & \sqrt{2} & 0 & \sqrt{2} & 0 \\
              0 & 0 & 0 & \sqrt{2} & 0 & 0 & 0 & \sqrt{2} & 0 \\
              \sqrt{2} & 0 & 0 & 0 & \sqrt{2} & 0 & 0 & 0 & \sqrt{2} \\
              0 & \sqrt{2} & 0 & 0 & 0 & \sqrt{2} & 0 & 0 & 0
      \end{pmatrix}
 \end{equation*}

Der dazugehörige Vektor der Ausgangsintensitäten $I_4$ und der Vektor mit den Intensitäten der
Projektionen bestehen aus:

\begin{equation*}
  \vec{I_4} = \begin{pmatrix}
      6.65 \pm 0.12 \\
      4.83 \pm 0.10  \\
      4.86 \pm 0.10  \\
      8.81 \pm 0.14  \\
      7.86 \pm 0.13  \\
      9.52 \pm 0.14  \\
      7.99 \pm 0.13  \\
      3.21 \pm 0.08  \\
      5.52 \pm 0.11  \\
      13.22 \pm 0.17  \\
      3.38 \pm 0.8 \\
      5.36 \pm 0.11
  \end{pmatrix},\
   \vec{\tilde{I_4}} = \begin{pmatrix}
   3.15 \pm 0.38 \\
   3.48 \pm 0.46 \\
   3.47 \pm 0.45 \\
   2.88 \pm 0.33 \\
   2.99 \pm 0.36 \\
   2.80 \pm 0.32 \\
   -1.60 \pm 1.06 \\
   3.89 \pm 0.56 \\
   -1.22 \pm 1.09 \\
   -2.10 \pm 1.04 \\
   3.84 \pm 0.54 \\
   -1.1 \pm 1.10
 \end{pmatrix}
\end{equation*}

Analog zu den Würfeln 2 und 3 wird auch hier die Methode der kleinsten Quadrate
angewandt. Mit den Ergebnissen und dem normalisierten Logarithmus ergeben sich
folgende Werte für die Absorptionskoeffizienten:

\begin{table}
  \centering
  \caption{Bestimmte Absorptioinskoeffizienten aus den verschiedenen Projektionen für Würfel 4.}
  \label{tab:5}
  \begin{tabular}{c
                  S[table-format=1.4(1)]}
                  \toprule
                  {Teilwürfel} & {Absorptionskoeffizient $mu$, $\si{\per\centi\meter}$}\\
                  \midrule
                  1 & 1.137(363) \\
                  2 & 0.608(414) \\
                  3 & 0.938(369) \\
                  4 & 0.264(407) \\
                  5 & 1.322(234) \\
                  6 & 0.872(412) \\
                  7 & 1.267(361) \\
                  8 & 0.360(412) \\
                  9 & 0.991(369) \\
  \end{tabular}
\end{table}
