\section{Zielsetzung}
\label{sec:Zielsetzung}

Ziel des Versuchs ist die Bestimmung der Materialien von neun Würfeln in einem
Aluminiumgehäuse anhand ihrer Absorptionskoeffizienten.

\section{Theorie}
\label{sec:Theorie}

% Tomographie
Tomographie ist ein Verfahren, bei welchem aus verschiedenen
Querschnittsbildern ein Objekt räumlich abgebildet wird. In diesem Versuch
wird dazu $\gamma$-Strahlung verwendet.
Trifft $\gamma$-Strahlung auf Materie, verringert sich die Intensität $N$
nach der Strecke $r$ von $I_0$ auf
\begin{equation*}
  N = N_0 \exp\left(-\mu r\right)
\end{equation*}
mit dem Absorptionskoeffizienten $\mu$. Wird eine Probe bestehend aus
neun Würfeln $i$ der Dicke $d_\text{i}$ bestrahlt, lässt sich
die Gleichung zu
\begin{equation*}
  \sum_\text{i} \mu_\text{i} d_\text{i} = \ln\left(\frac{I_0}{N_\text{j}}\right)
\end{equation*}
umstellen, dabei ist $N_\text{j}$ die Ausgangsintensität der j-ten Messung.
Diese Umstellung lässt sich ebenfalls als Matrixgleichung der
Form
\begin{equation}
  \begin{aligned}
    \mathbf{A} \cdot \vec{\mu} &= \vec{I} \text{ mit} \\
    \vec{\mu} &= (\mu_1,...,\mu_9)^\text{T} \text{ und} \\
    \vec{I} &= \ln(\sfrac{I_0}{N_\text{j}})
  \end{aligned}
  \label{eqn:matrixform}
\end{equation}
realisieren.
Der Vektor $\vec{\mu}$ beinhaltet die
Absorptionskoeffizienten,
der Vektor $\vec{I}$ stellt die rechte Seite
der Gleichung dar und die Matrix $\mathbf{A}$ beschreibt die Würfelgeometrie.
Werden mehr Schichten aufgenommen, als Würfel in der Probe sind, lassen sich
die Absorptionskoeffizienten mittels
\begin{equation}
  \vec{\mu} = \left(\mathbf{A}^\text{T} \mathbf{A}\right)^{-1} \cdot
  \mathbf{A}^\text{T} \vec{I}
  \label{eqn:least-squares}
\end{equation}
mit den zugehörigen Abweichungen
\begin{equation}
  \sigma_\text{i} =
  \sqrt{\text{diag}\left\{\left(\mathbf{A}^\text{T} \mathbf{A}\right)^{-1}\right\}}
  \label{eqn:least-squares-error}
\end{equation}
nach der Methode der kleinsten Quadrate bestimmen.

% Gamma-Spektrum
Wird die Intensität von Strahlung nach Durchlauf einer Probe gegen die Wellenlänge
aufgetragen, ergibt sich ein typischer Verlauf, welcher aus den unterschiedlichen
Wechselwirkungen der Strahlung im Material resultiert.
Für große Wellenlängen dominiert der Comptoneffekt,
welcher zu einer kontinierlichen Absorptionslinie führt,
die bei einer bestimmten Wellenlänge abrupt in der Intensität sinkt.
Diese sogenannte Comptonkante befindet sich bei der Wellenlänge,
bei welcher ein einfallendes Photon einen maximalen
Energieübertrag an das Elektron leistet.
Bei etwas höheren Wellenlängen ist eine Spitze messbar,
welche auf dem Photoeffekt beruht, der sogenannte \enquote{Photopeak}.
Bei noch kleineren Wellenlängen steigt die Intensität wieder an,
da benötigte Mindestenergie für die Paarerzeugung erreicht ist.

% \begin{figure}
%   \centering
%   \includegraphics[height=8.0cm]{content/Bild.png}
%   \caption{Bilduterschrift}
%   \label{fig:Bild}
% \end{figure}
