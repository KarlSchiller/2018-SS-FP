\section{Durchführung}
\label{sec:Durchführung}

\subsection{Versuchsaufbau}
Der Pumpstand ist wie in Abbildung~\ref{fig:Aufbau} gezeigt aufgebaut.
Hier ist es wichtig, möglichst sorgfältig zu arbeiten,
da sonst Fehler durch Lecks an Gummidichtungen und Flanschen entstehen.
\begin{figure}
  \begin{subfigure}[c]{0.5\textwidth}
    \includegraphics[width=\textwidth]{IMG_6888.JPG}
  \end{subfigure}
  \begin{subfigure}[c]{0.5\textwidth}
    \includegraphics[width=\textwidth]{IMG_6889.JPG}
  \end{subfigure}
  \caption{Aufbau des Pumpstandes und Beschriftung der Einzelteile.}
  \label{fig:Aufbau}
\end{figure}

Dabei wurden diejenigen Teile verwendet,
die aus Tabelle~\ref{tab:Teile} bzw. Abbildung~\ref{fig:Teile} zu entnehmen sind.

\begin{figure}
  \centering
  \includegraphics[width=0.4\textwidth]{IMG_6891.JPG}
  \caption{Verwendete Teile zum Aufbau des Pumpstandes.}
  \label{fig:Teile}
\end{figure}

\begin{table}
  \caption{Verwendete Bauteile zum Aufbau des zur Messung verwendeten Pumpstandes~\cite{anleitung}.}
  \label{tab:Teile}
  \begin{tabular}{c c | c c}
    \toprule
    Abkürzung & Bezeichnung/Name & Abkürzung & Bezeichnung/Name \\
    \midrule
    S1 & Schlauch 1 & D1 & Nadel-/ Dosierventil 1 \\
    S2 & Schlauch 2 & P1 & analoges Pirani Druckmessgerät \\
    B1 & großer Rezipient & B2 & Bauteil \\
    B3 & Bauteil & B4 & Bauteil \\
    B5 & Bauteil & B6 & Bauteil \\
    V1 & Klappenventil & V2 & Kugelventil 2 \\
    V3 & Kugelventil 3 & V4 & Kugelventil 4 \\
  \end{tabular}
\end{table}

Zuerst wird der Pumpstand mit der Drehschieberpumpe evakuiert.
Nach ungefähr \SI{10}{\minute}
stellt sich ein Druck von $\SI{5e-3}{\milli\bar}$ am Pirani-Messgerät ein und die TMP
kann durch Öffnen von V1 und Schließen von V2 hinzugeschaltet werden.
Jetzt wird das Glühkathoden Vakuummeter eingeschaltet.
Im Anschluss wird der Rezipient ca. \SI{10}{\minute} mit dem Heißluftföhn erhitzt.
Das führt dazu, dass Wasserablagerungen im Rezipienten verdampfen
und so die Desorptionsrate nach der Abkühlung geringer ist.
Wird am Glühkathodenvakuummeter ein Enddruck zwischen $\SIrange{2e-5}{8e-5}{\milli\bar}$ erreicht,
so gilt der Aufbau als dicht und es kann mit den Messungen begonnen werden.

% Messungen
\subsection{Evakuierungskurve der Drehschieberpumpe}
Zuerst wird die TMP durch das Schließen von V1 und V5 abgeschiebert und V2 geöffnet.
Bei laufender Drehschieberpumpe werden nun die Belüftungsventile D1 und V3 geöffnet,
um den Rezipienten auf Atmosphärendruck zu bringen. Daraufhin werden die Belüftungsventile
wieder geschlossen und der Druckabfall gegen die Zeit am Pirani-Messgerät aufgenommen.
Diese Messung wird 5-mal durchgeführt.

\subsection{Leckratenmessung der Drehschieberpumpe}
Bei laufender Drehschieberpumpe wird mit dem Dosierventil D1 ein Gleichgewichtsdruck $p_\text{G}$
im Rezipienten erzeugt und nach Abschiebern von V4 der Druckanstieg mit der Zeit durch
das Pirani-Messgerät aufgenommen. Dies Messung wird für die Gleichgewichtsdrücke
$p_\text{G}$ = \{$\SI{0.1}{\milli\bar}$, $\SI{0.4}{\milli\bar}$, $\SI{0.8}{\milli\bar}$,
$\SI{1.0}{\milli\bar}$\} jeweils 3-mal durchgeführt.
Bevor nun die Messungen an der TMP durchgeführt werden können, muss diese wieder
hinzugeschaltet werden. Dazu wird V1 und V5 geöffnet und V2 geschlossen. Wenn der
Druck durch die Drehschieberpumpe unter $\SI{5e-3}{\milli\bar}$ gefallen ist, kann
die TMP eingeschaltet werden.

\subsection{Evakuierungsmessung der Turbomolukularpumpe}
Bei laufender TMP wird mit dem Dosierventil D1 ein Druck von $\SI{5e-3}{\milli\bar}$ eingestellt
und anschließend D1 und V3 möglichst zeitgleich geschlossen. Der auftretende Druckabfall
wird mit der Zeit durch ein Glühkathoden-Vakuummeter aufgenommen. Diese Messung
wird analog zu der bei der Drehschieberpumpe 5-mal wiederholt.

\subsection{Leckratenmessung der Turbomolukuralpumpe}
Analog zur Leckratenmessung  der Drehschieberpumpe wird auch hier mit Hilfe des Nadelventils
D1 ein Gleichgewichtsdruck $p_\text{G}$ eingestellt und dann V1 geschlossen. Daraufhin
kann der Druckanstieg über das Glühkathoden-Vakuummeter und die Zeit aufgenommen werden.
Diese Messung wird bei $p_\text{G}$ = \{$\SI{5e-5}{\milli\bar}, \SI{10e-5}{\milli\bar},
\SI{150e-5}{\milli\bar}, \SI{20e-5}{\milli\bar}$\} jeweils 3-mal durchgeführt und der Druckanstieg mit einem
Glühkathoden-Vakuummeter aufgenommen.

\subsection{Volumenbestimmung}
Da sich das Volumen des Rezipienten aufgrund von geschlossenen Ventilen und Verbindungsstücken
bei jeder Messreihe ändert, muss für jede Messung das Gesamtvolumen bestimmt werden. Dazu werden
die verwendeten Teile nach dem Abbau des Pumpstandes mit Lineal und Schieblehre ausgemessen
und bei der Volumenberechnung mit einbezogen.
