\newpage
\section{Durchführung}
\label{sec:Durchführung}

\subsection{Versuchsaufbau}
Der Pupmstand ist wie in Abbildung \ref{fig:Aufbau} aufgebaut. Hier ist es wichtig
möglichst genau zu arbeiten, da sonst Fehler durch Lecks an Gummis und Flanschen
entstehen.

Zuerst wird der Pumpstand mit der Drehschieberpumpe evakuiert. Nach ungefähr 10$\,$min
stellt sich ein Druck von $\SI{5e-2}{\milli\bar}$ ein und die TMP kann hinzugeschaltet
werden.
Jetzt kann das Glühkathoden Vakuummeter eingeschaltet werden. Nun kann der Rezipient
ca. 10$\,$min der Rezipient mit dem Heißluftföhn erhitzt werden. Das führt dazu,
dass Wasserablagerungen im Rezipienten verdampfen und so die Desorptionsrate abnimmt.
Wird ein Enddruck zwischen $\SIrange{2e-5}{8e-5}{\milli\bar}$ erreicht, so gilt der
Aufbau als dicht und es kann mit den Mesungen begonnen werden.

\subsection{Messungen}
Damit die Messung der Drehschieberpumpedurchgeführt werden kann, muss zuerst das
Glühkathoden Vakuummeter abgeschaltet werden und die TMP am Ein- und Ausgang
abgeschiebert werden.
Nun kann das Ventil 5 geöffnet werden und die Messung beginnen. Dabei sollte der
Druck nach der Belüftung bei $\SI{1000}{\milli\bar}$. Bei einer Leckratenmessung
liegt ein Gleichgewichtsruck zwischen $\SIrange{0.1}{1000}{\milli\bar}$ vor.

% \begin{figure}
%   \centering
%   \includegraphics[height=8.0cm]{content/Versuchsaufbau1.png}
%   \caption{Bildunterschrift}
%   \label{fig:Versuchsaufbau1}
% \end{figure}
