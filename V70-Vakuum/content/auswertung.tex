\newpage
\section{Auswertung}
\label{sec:Auswertung}

\subsection{Berechnung des Rezipientenvolumens}
\label{sec:Volumina}

Für die Bestimmung der Pumpensaugfähigkeit ist das Volumens des Rezipienten
vonnöten. Dazu sind in den Tabellen \ref{tab:vol1} und \ref{tab:vol2} die
in dem Kapitel zum Versuchsaufbau erläuterten Bauteile samt ihres Volumens aufgeführt.
Die Abkürzung \enquote{D} steht dabei für die Verwendung
des jeweiligen Bauteils bei der Vermessung der
Drehschieberpumpe und die Akürzung \enquote{T} für die Turbomolekularpumpe.

% Aus Anleitung übernommene Volumina
\begin{table}
    \centering
    \caption{Verwendete Bauteile mit nicht selbst vermessenen Volumina \cite[p.~5]{anleitung}.}
    \label{tab:vol1}
    \sisetup{table-format=1.3}
    \begin{tabular}{c
                    S @{${}\pm{}$} S
                    c}
    \toprule
        {Bauteil} &
        \multicolumn{2}{c}{$V\:/\:\si{\liter}$} &
        {Verwendung} \\
    \midrule
    B1 & 9.5 & 0.8 & D,T \\
    B2 & 0.177 & 0.09 & D,T \\
    B3 & 0.25 & 0.01 & D,T \\
    B4 & 0.067 & 0.004 & T\\
    B5 & 0.016 & 0.002 & D \\
    V1 auf & 0.044 & 0.004 & T \\
    V1 zu & 0.022 & 0.002 & D \\
    V2 auf & 0.015 & 0.002 & D \\
    V2 zu & 0.005 & 0.001 & T \\
    V3 auf & 0.015 & 0.002 & D,T \\
    V3 zu & 0.005 & 0.001 & D,T \\
    V4 & 0.025 & 0.005 & D \\
    V5 zu & 0.005 & 0.001 & D \\
    \end{tabular}
\end{table}


In Tabelle \ref{tab:vol2} sind neben den Volumina auch die gemessenen
Durchmesser $d$ und die Längen $l$ der Bauteile aufgeführt, aus diesen wurde
das jeweilige Volumen unter Annahme einer Zylindergeometrie nach
\begin{equation*}
  V = \pi \left(\frac{d}{2}\right) l
\end{equation*}
mit dem zugehörigen Fehler
\begin{equation*}
  \Delta V =
  \sqrt{\left(\frac{\partial V}{\partial d} \;\Delta d\right)^2 +
  \left(\frac{\partial V}{\partial l} \;\Delta l\right)^2} =
  \pi \left(\frac{d}{2}\right) \sqrt{
  \left(l \Delta d \right)^2 +
  \left( \frac{d}{2} \Delta l\right)^2}
\end{equation*}
bestimmt.

% Selbst vermessene Volumina
\begin{table}
    \centering
    \caption{Verwendete Bauteile mit selbst vermessenen Volumina.}
    \label{tab:vol2}
    \sisetup{table-format=2.1}
    \begin{tabular}{c
                    S @{${}\pm{}$} S[table-format=1.1]
                    S[table-format=4.1] @{${}\pm{}$} S
                    S[table-format=1.4] @{${}\pm{}$} S[table-format=1.4]
                    c}
    \toprule
        {Bauteil} &
        \multicolumn{2}{c}{$d\:/\:\si{\milli\meter}$} &
        \multicolumn{2}{c}{$l\:/\:\si{\milli\meter}$} &
        \multicolumn{2}{c}{$V\:/\:\si{\liter}$} &
        {Verwendung} \\
    \midrule
    S1 & 16 & 1 & 430 & 10 & 0.086 & 0.011 & D \\
    S2 & 24.8 & 0.1 & 1210 & 20 & 0.582 & 0.011 & D \\
    E1 & 12.0 & 0.1 & 59.9 & 0.1 & 0.0067 & 0.0001 & D \\
    B6 & 40.5 & 0.1 & 130.0 & 0.1 & 0.225 & 0.007 & D,T \\
    \end{tabular}
\end{table}


Das Bauteil B6 ist ein T-Stück, hierfür wurde zum angegebenen Volumen des Zylinders
in Tabelle \ref{tab:vol2}
das Volumen des T-Teils addiert. Dieses hatte ebenfalls einen Durchmesser
von \SI{40.5(1)}{\milli\meter} und eine Länge von \SI{4.5(5)}{\centi\meter}.
Eine weitere Besonderheit ist das Handventil V3, dieses wurde für die Messung
der Evakuierungskurven geschlossen und für die Messung der Leckratenmessung
geschlossen. Diese Volumenänderung ist beim Rezipientenvolumen jedoch kleiner
als dessen Unsicherheit und
wurde deshalb im Folgenden vernachlässigt.

Für die beiden Pumpen ergibt sich ein Rezipientenvolumen von
\begin{align*}
  V_\text{Dreh} &= \SI{10.9(8)}{\liter} \text{ beziehungsweise}\\
  V_\text{Turbo} &= \SI{10.3(8)}{\liter}.
\end{align*}


\subsection{Evakuierungskurve der Drehschieberpumpe}
\label{sec:AuswEvaD}

- Zeiten ohne Fehler gemittelt, da nicht sinnvoll angebbar
- druckfehler 0.2
- druck umgeformt mit dem ln und Fehlerfortpflanzung
- linearer Fit mit scipy
- aus Steigung und Volumen Saugvermögen
- Alles in einen Plot, 3 Saugvermögen angeben


\subsection{Evakuierungskurve der Turbomolekularpumpe}
\label{sec:AuswEvaT}

\subsection{Leckratenmessung der Drehschieberpumpe}
\label{sec:AuswLeckD}

\subsection{Leckratenmessung der Turbomolekularpumpe}
\label{sec:AuswLeckT}

% \begin{figure}
%   \centering
%   \includegraphics[width=\textwidth]{Plot.pdf}
%   \caption{Bildunterschrift}
%   \label{fig:Plot1}
% \end{figure}
