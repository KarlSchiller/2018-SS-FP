\newpage
\section{Auswertung}
\label{sec:Auswertung}

\subsection{Berechnung des Rezipientenvolumens}
\label{sec:Volumina}

Für die Bestimmung der Pumpensaugfähigkeit ist das Volumens des Rezipienten
vonnöten. Dazu sind in den Tabellen \ref{tab:vol1} und \ref{tab:vol2} die
in dem Kapitel zum Versuchsaufbau erläuterten Bauteile samt ihres Volumens aufgeführt.
Die Abkürzung \enquote{D} steht dabei für die Verwendung
des jeweiligen Bauteils bei der Vermessung der
Drehschieberpumpe und die Akürzung \enquote{T} für die Turbomolekularpumpe.

% Aus Anleitung übernommene Volumina
\begin{table}
    \centering
    \caption{Verwendete Bauteile mit nicht selbst vermessenen Volumina \cite[p.~5]{anleitung}.}
    \label{tab:vol1}
    \sisetup{table-format=1.3}
    \begin{tabular}{c
                    S @{${}\pm{}$} S
                    c}
    \toprule
        {Bauteil} &
        \multicolumn{2}{c}{$V\:/\:\si{\liter}$} &
        {Verwendung} \\
    \midrule
    B1 & 9.5 & 0.8 & D,T \\
    B2 & 0.177 & 0.09 & D,T \\
    B3 & 0.25 & 0.01 & D,T \\
    B4 & 0.067 & 0.004 & T\\
    B5 & 0.016 & 0.002 & D \\
    V1 auf & 0.044 & 0.004 & T \\
    V1 zu & 0.022 & 0.002 & D \\
    V2 auf & 0.015 & 0.002 & D \\
    V2 zu & 0.005 & 0.001 & T \\
    V3 auf & 0.015 & 0.002 & D,T \\
    V3 zu & 0.005 & 0.001 & D,T \\
    V4 & 0.025 & 0.005 & D \\
    V5 zu & 0.005 & 0.001 & D \\
    \end{tabular}
\end{table}


In Tabelle \ref{tab:vol2} sind neben den Volumina auch die gemessenen
Durchmesser $d$ und die Längen $l$ der Bauteile aufgeführt, aus diesen wurde
das jeweilige Volumen unter Annahme einer Zylindergeometrie nach
\begin{equation*}
  V = \pi \left(\frac{d}{2}\right) l
\end{equation*}
mit dem zugehörigen Fehler
\begin{equation*}
  \Delta V =
  \sqrt{\left(\frac{\partial V}{\partial d} \;\Delta d\right)^2 +
  \left(\frac{\partial V}{\partial l} \;\Delta l\right)^2} =
  \pi \left(\frac{d}{2}\right) \sqrt{
  \left(l \Delta d \right)^2 +
  \left( \frac{d}{2} \Delta l\right)^2}
\end{equation*}
bestimmt.

% Selbst vermessene Volumina
\begin{table}
    \centering
    \caption{Verwendete Bauteile mit selbst vermessenen Volumina.}
    \label{tab:vol2}
    \sisetup{table-format=2.1}
    \begin{tabular}{c
                    S @{${}\pm{}$} S[table-format=1.1]
                    S[table-format=4.1] @{${}\pm{}$} S
                    S[table-format=1.4] @{${}\pm{}$} S[table-format=1.4]
                    c}
    \toprule
        {Bauteil} &
        \multicolumn{2}{c}{$d\:/\:\si{\milli\meter}$} &
        \multicolumn{2}{c}{$l\:/\:\si{\milli\meter}$} &
        \multicolumn{2}{c}{$V\:/\:\si{\liter}$} &
        {Verwendung} \\
    \midrule
    S1 & 16 & 1 & 430 & 10 & 0.086 & 0.011 & D \\
    S2 & 24.8 & 0.1 & 1210 & 20 & 0.584 & 0.011 & D \\
    E1 & 12.0 & 0.1 & 59.9 & 0.1 & 0.0067 & 0.0001 & D \\
    B6 & 40.5 & 0.1 & 130.0 & 0.1 & 0.225 & 0.007 & D,T \\
    \end{tabular}
\end{table}


Das Bauteil B6 ist ein T-Stück, hierfür wurde zum angegebenen Volumen des Zylinders
in Tabelle \ref{tab:vol2}
das Volumen des T-Teils addiert. Dieses hatte ebenfalls einen Durchmesser
von \SI{40.5(1)}{\milli\meter} und eine Länge von \SI{4.5(5)}{\centi\meter}.
Eine weitere Besonderheit ist das Handventil V3, dieses wurde für die Messung
der Evakuierungskurven geschlossen und für die Messung der Leckratenmessung
geschlossen. Diese Volumenänderung ist beim Rezipientenvolumen jedoch kleiner
als dessen Unsicherheit und
wurde deshalb im Folgenden vernachlässigt.

Für die beiden Pumpen ergibt sich ein Rezipientenvolumen von
\begin{align*}
  V_\text{Dreh} &= \SI{10.9(8)}{\liter} \text{ beziehungsweise}\\
  V_\text{Turbo} &= \SI{10.3(8)}{\liter}.
\end{align*}


\subsection{Evakuierungskurve der Drehschieberpumpe}
\label{sec:AuswEvaD}

Die gemessenen Zeiten $t_\text{1..6}$, an denen die Druckwerte $p$ erreicht
wurden, sind in Tabelle \ref{tab:evak_D} aufgeführt. Dabei wurden die Drücke
mit einem digitalen Pirani Messgerät mit einer
Unsicherheit von \SI{20}{\percent} aufgenommen. Für die gemessenen Zeiten lässt
sich nur sehr schwer eine experimentelle Unsicherheit angeben, zumal diese stark
zwischen Stützstellen variieren müsste. Für die weitere Auswertung wurden die
Zeiten gemittelt, dieser Mittelwert $\mean{t_\text{1..6}}$ ist ebenfalls in
Tabelle \ref{tab:evak_D} aufgeführt.
Dabei wurden Mittelwert und empirische Standardabweichung nach
\begin{align*}
  \mean{t_\text{1..6}} &= \frac{1}{6} \sum_{i = 1}^6 t_\text{i} \\
  \Delta t_\text{1..6} &=
  \sqrt{\frac{1}{5} \sum_{i = 1}^6 \left(t_\text{i} - \mean{t_\text{1..6}}\right)^2}
\end{align*}
berechnet.

\begin{table}
    \centering
    \caption{Drücke und gemessene Zeiten der $p\left(t\right)$-Evakuierungsmessung der Drehschieberpumpe.}
    \label{tab:evak_D}
    \sisetup{table-format=3.2}
    \begin{tabular}{S[table-format=4.2] @{${}\pm{}$} S[table-format=3.2]
                    S[table-format=2.2] @{${}\pm{}$} S[table-format=1.2]
                    S S S S S S
                    S[table-format=3.1] @{${}\pm{}$} S[table-format=1.1]}
    \toprule
        \multicolumn{2}{c}{$p\:/\:\si{\milli\bar}$} &
        \multicolumn{2}{c}{$\ln\left(\frac{p(t)-p_\text{e}}{p_0-p_\text{e}}\right)$} &
        \multicolumn{6}{c}{$t_\text{1..6}\:/\:\si{\second}$} &
        \multicolumn{2}{c}{$\mean{t_\text{1..6}}\:/\:\si{\second}$} \\
    \midrule
    1013.0 & 5.0 & 0.01 & 0.2 & 0.0 & 0.0 & 0.0 & 0.0 & 0.0 & 0.0 & 0.0 & 0.0 \\
    100.0 & 20.0 & -2.3 & 0.28 & 15.28 & 16.48 & 14.60 & 14.03 & 13.89 & 13.52 & 14.6 & 0.4 \\
    60.0 & 12.0 & -2.81 & 0.28 & 26.21 & 25.90 & 25.86 & 26.07 & 25.66 & 25.57 & 25.9 & 0.1 \\
    40.0 & 8.0 & -3.22 & 0.28 & 32.45 & 32.04 & 32.80 & 32.57 & 33.26 & 32.57 & 32.6 & 0.2 \\
    20.0 & 4.0 & -3.91 & 0.28 & 40.45 & 40.51 & 40.84 & 40.89 & 40.79 & 41.05 & 40.8 & 0.1 \\
    10.0 & 2.0 & -4.61 & 0.28 & 48.04 & 47.88 & 48.47 & 48.46 & 48.62 & 48.82 & 48.4 & 0.1 \\
    8.0 & 1.6 & -4.83 & 0.28 & 50.28 & 50.27 & 50.73 & 50.92 & 50.81 & 51.11 & 50.7 & 0.1 \\
    6.0 & 1.2 & -5.12 & 0.28 & 53.18 & 53.28 & 53.99 & 54.05 & 54.12 & 54.23 & 53.8 & 0.2 \\
    4.0 & 0.8 & -5.53 & 0.28 & 56.71 & 56.93 & 57.52 & 57.87 & 57.88 & 57.97 & 57.5 & 0.2 \\
    2.0 & 0.4 & -6.23 & 0.29 & 63.57 & 63.39 & 64.20 & 64.51 & 64.24 & 64.70 & 64.1 & 0.2 \\
    1.0 & 0.2 & -6.94 & 0.29 & 70.89 & 70.90 & 71.81 & 72.17 & 72.13 & 72.26 & 71.7 & 0.3 \\
    0.8 & 0.16 & -7.17 & 0.29 & 73.60 & 73.56 & 74.61 & 74.60 & 74.43 & 75.12 & 74.3 & 0.3 \\
    0.6 & 0.12 & -7.47 & 0.29 & 77.63 & 77.92 & 79.00 & 79.10 & 79.09 & 79.33 & 78.7 & 0.3 \\
    0.4 & 0.08 & -7.9 & 0.29 & 84.65 & 84.85 & 86.17 & 86.13 & 86.25 & 86.89 & 85.8 & 0.4 \\
    0.2 & 0.04 & -8.68 & 0.31 & 99.95 & 98.72 & 100.90 & 101.53 & 101.48 & 102.26 & 100.8 & 0.5 \\
    0.1 & 0.02 & -9.57 & 0.36 & 115.90 & 112.91 & 116.61 & 117.03 & 116.95 & 118.55 & 116.3 & 0.8 \\
    0.08 & 0.02 & -9.9 & 0.4 & 123.62 & 119.93 & 124.87 & 125.65 & 125.39 & 127.28 & 124.5 & 1.0 \\
    0.06 & 0.01 & -10.41 & 0.49 & 141.80 & 134.59 & 143.81 & 144.75 & 143.08 & 146.55 & 142.4 & 1.7 \\
    \end{tabular}
\end{table}


Des Weiteren lässt sich Gleichung \eqref{eqn:druck} zu
\begin{equation*}
  f\left(p\right) = \ln\left(\frac{p(t)-p_\text{e}}{p_0-p_\text{e}}\right) =
  - \frac{S}{V} \;t
\end{equation*}
umformen. Dieser Zusammenhang motiviert die in Abbildung \ref{fig:evak_D}
dargestellten linearen Regressionen an die Messwerte. Die verwendeten Messwerte
sind ebenfalls in Tabelle \ref{tab:evak_D} aufgeführt, dabei ergaben sich die
Unsicherheiten nach Gaußscher Fehlerfortpflanzung zu
\begin{align*}
  \Delta f &=
  \sqrt{\left(\frac{\partial f}{\partial p} \Delta p\right)^2 +
  \left(\frac{\partial f}{\partial p_\text{e}} \Delta p_\text{e}\right)^2 +
  \left(\frac{\partial f}{\partial p_\text{0}} \Delta p_\text{0}\right)^2} \\
  &=
  \sqrt{\left(\frac{\Delta p}{p - p_\text{e}}\right)^2 +
  \left(\frac{\Delta p_0}{p_0 - p_\text{e}}\right)^2 +
  \left(\frac{\Delta p_\text{e}}{p_0 - p_\text{e}} -
  \frac{\Delta p_\text{e}}{p - p_\text{e}}\right)^2}.
\end{align*}
Der Startdruck $p_0$ betrug \SI{1000}{\milli\bar} und der Enddruck $p_\text{e}$
betrug \SI{0.03}{\milli\bar}.

\begin{figure}
  \centering
  \includegraphics[width=\textwidth]{build/evak_D.pdf}
  \caption{Evakuierungskurve der Drehschieberpumpe samt linearer Regression.}
  \label{fig:evak_D}
\end{figure}

Die in Abbildung \ref{fig:evak_D} dargestellte linearen Regressionen wurden
mit der Funktion \texttt{curve\_fit} des Pakets \texttt{scipy.optimze} \cite{scipy}
erstellt. Aus den erhaltenen Steigungen $m$ lassen sich die Saugvermögen der Pumpe
nach
\begin{align*}
    S &= - m \; V_\text{Dreh} \\
    \Delta S &=
    \sqrt{\left(\frac{\partial S}{\partial m} \Delta m\right)^2 +
    \left(\frac{\partial S}{\partial V_\text{Dreh}} \Delta V_\text{Dreh}\right)^2} \\
    &=
    \sqrt{\left(\Delta m \; V_\text{Dreh}\right)^2 +
    \left(m \Delta \; V_\text{Dreh}\right)^2}
\end{align*}
berechnen. Sie sind in Tabelle \ref{tab:saug_evak_D} dargestellt.

\begin{table}
    \centering
    \caption{Steigungen und Saugvermögen der Regressionen an die Evakuierungsmessung der Drehschieberpumpe.}
    \label{tab:saug_evak_D}
    \sisetup{table-format=1.2}
    \begin{tabular}{c
                    S[table-format=2.3] @{${}\pm{}$} S[table-format=1.3]
                    S @{${}\pm{}$} S}
    \toprule
        {Regression} &
        \multicolumn{2}{c}{$m\:/\:\si{\raiseto{-1}\second}$} &
        \multicolumn{2}{c}{$S\:/\:\si{\liter\raiseto{-1}\second}$} \\
    \midrule
    blau & -0.050 & 0.004 &  0.55 & 0.06 \\
    grün & -0.099 & 0.001 & 1.08 & 0.08 \\
    rot & -0.054 & 0.001 & 0.59 & 0.04\\
    \end{tabular}
\end{table}

\FloatBarrier


\subsection{Evakuierungskurve der Turbomolekularpumpe}
\label{sec:AuswEvaT}

Analog zur Evakuierungskurve der Drehschieberpumpe wurde die Evakuierungskurve
der Turbomolekularpumpe ausgewertet. In Tabelle \ref{tab:evak_T} sind die
Stützstellen des Druckes $p$ samt der gemessenen Zeiten $t_\text{1..6}$
dargestellt. Der Mittelwert $\mean{t_\text{1..6}}$ und die logarithmierten
Druckwerte sind nach den selben Formeln wie die Werte der Drehschieberpumpe
berechnet. Die Ungenauigkeit der Glühkathode beträgt \SI{10}{\percent}, der
Startdruck beträgt \SI{5e-3}{\milli\bar} und der Enddruck beträgt
\SI{1.7e-5}{\milli\bar}.

\begin{table}
    \centering
    \caption{Drücke und gemessene Zeiten der $p\left(t\right)$-Evakuierungsmessung der Turbomolekularpumpe.}
    \label{tab:evak_T}
    \sisetup{table-format=2.2}
    \begin{tabular}{c
%                    S[table-format=1e1, scientific-notation = fixed, fixed-exponent = -6]
%                    @{${}\pm{}$}
%                    S[table-format=1e1, scientific-notation = fixed, fixed-exponent = -6]
                    S[table-format=1.2] @{${}\pm{}$} S[table-format=1.2]
                    S S S S S S
                    S[table-format=2.1] @{${}\pm{}$} S[table-format=1.1]}
    \toprule
%        \multicolumn{2}{c}{$p\:/\:\si{\milli\bar}$} &
        {$p\:/\:\si{\milli\bar}$} &
        \multicolumn{2}{c}{$\ln\left(\frac{p(t)-p_\text{e}}{p_0-p_\text{e}}\right)$} &
        \multicolumn{6}{c}{$t_\text{1..6}\:/\:\si{\second}$} &
        \multicolumn{2}{c}{$\mean{t_\text{1..6}}\:/\:\si{\second}$} \\
    \midrule
    $(5 \pm 0.5) \cdot 10^{-3}$ & 0.0 & 0.14 & 0.0 & 0.0 & 0.0 & 0.0 & 0.0 & 0.0 & 0.0 & 0.0 \\
    $(2 \pm 0.2) \cdot 10^{-3}$ & -0.92 & 0.14 & 0.76 & 0.80 & 0.70 & 0.63 & 0.62 & 0.85 & 0.7 & 0.1 \\
    $(8 \pm 0.8) \cdot 10^{-3}$ & -1.85 & 0.14 & 1.99 & 1.93 & 1.83 & 1.72 & 1.76 & 1.92 & 1.9 & 0.1 \\
    $(4 \pm 0.4) \cdot 10^{-4}$ & -2.57 & 0.14 & 2.68 & 2.78 & 2.36 & 2.61 & 2.51 & 2.81 & 2.6 & 0.2 \\
    $(2 \pm 0.2) \cdot 10^{-4}$ & -3.30 & 0.15 & 3.78 & 3.64 & 3.68 & 3.50 & 3.52 & 3.64 & 3.6 & 0.1 \\
    $(8 \pm 0.8) \cdot 10^{-5}$ & -4.37 & 0.16 & 5.38 & 5.31 & 5.14 & 5.14 & 5.01 & 5.30 & 5.2 & 0.1 \\
    $(6 \pm 0.6) \cdot 10^{-5}$ & -4.75 & 0.18 & 6.11 & 6.04 & 5.97 & 5.90 & 5.78 & 6.00 & 6.0 & 0.1 \\
    $(4 \pm 0.4) \cdot 10^{-5}$ & -5.38 & 0.21 & 8.19 & 7.36 & 7.31 & 7.39 & 6.93 & 7.27 & 7.4 & 0.4 \\
    $(3 \pm 0.3) \cdot 10^{-5}$ & -5.95 & 0.28 & 11.03 & 10.30 & 9.67 & 10.37 & 9.24 & 9.36 & 10.0 & 0.7 \\
    \end{tabular}
\end{table}


\begin{figure}
  \centering
  \includegraphics[width=\textwidth]{build/evak_T.pdf}
  \caption{Evakuierungskurve der Turbomolekularpumpe samt linearer Regression.}
  \label{fig:evak_T}
\end{figure}

Wiederum wurden an die logarithmierten Drücke und gemittelten Zeiten
lineare Regressionen mit der Funktion \texttt{curve\_fit} durchgeführt.
Diese Regressionen sind in Abbildung \ref{fig:evak_T} dargestellt.
Die sich ergebenden Steigungen und daraus berechneten Saugvermögen sind in
Tabelle \ref{tab:saug_evak_T} dargestellt. Dabei wurde an die Steigung das
Volumen des Rezipienten beim Betreiben der Turbomolekularpumpe $V_\text{Turbo}$
multipliziert.

\begin{table}
    \centering
    \caption{Steigungen und Saugvermögen der Regressionen an die Evakuierungsmessung der Turbomolekularpumpe.}
    \label{tab:saug_evak_T}
    \sisetup{table-format=1.1}
    \begin{tabular}{c
                    S[table-format=1.2] @{${}\pm{}$} S[table-format=1.2]
                    S @{${}\pm{}$} S}
    \toprule
        {Regression} &
        \multicolumn{2}{c}{$m\:/\:\si{\raiseto{-1}\second}$} &
        \multicolumn{2}{c}{$S\:/\:\si{\liter\raiseto{-1}\second}$} \\
    \midrule
    blau & -0.90 & 0.05 & 9.2 & 0.9 \\
    grün & -0.32 & 0.05 & 3.3 & 0.5 \\
    \end{tabular}
\end{table}



\subsection{Leckratenmessung der Drehschieberpumpe}
\label{sec:AuswLeckD}

\subsection{Leckratenmessung der Turbomolekularpumpe}
\label{sec:AuswLeckT}
