\section{Diskussion der Ergebnisse}
\label{sec:Diskussion}

% TODO Neue Ergebnisse des Saugvermögens diskutieren

% Umschalten Druckbereich
Die Messung des Druckes erfolgte bei der Drehschieberpumpe mit einem
Pirani-Messgerät und bei der Turbomolekularpumpe mit einer Glühkathode mit der
linearen Skala. Dabei variierten die Drücke über verschiedene Größenordnungen,
sodass der Messbereich während der Messungen jeweils umgestellt werden musste.

% Unsicherheit der Zeiten
Bei der Auswertung wurde keine Unsicherheit der gemessenen Zeiten verwendet.
Dies liegt daran, dass sie sich nicht eindeutig bestimmen lässt.
Aufgrund des oben beschriebenen Umschaltens des Messbereiches der Druckmesser konnten
einige Zeiten genauer als andere aufgenommen werden.
Da diese Zeiten jedoch per Hand mit einer digitalen Stoppuhr gemessen wurden,
liegt sehr wahrscheinlich ein unbestimmter systematischer Fehler aufgrund einer
nicht verschwindenden Reaktionszeit des Menschen vor.
Deshalb sind die eigentlichen Unsicherheiten der Zeiten und somit auch
Steigungen der Regressionen und sich ergebende Saugvermögen größer als
in der Auswertung angegeben.
% TODO Warum größer?

% Vergleich der Saugvermögen
In den Abbildungen \ref{fig:vgl_D} und \ref{fig:vgl_T} sind die verschiedenen
bestimmten Saugvermögen der Drehschieberpumpe beziehungsweise der
Turbomolekularpumpe eingezeichnet.
Dabei ist die mit \enquote{Hersteller} gekennzeichnete Linie eine Herstellerangabe
aus der Versuchsanleitung \cite{anleitung} von \SI{1.1}{\liter\raiseto{-1}\second}
für die Drehschieberpumpe und \SI{77}{\liter\raiseto{-1}\second} für die
Turbomolekularpumpe. Hier ist weder ein Druckbereich, noch eine Unsicherheit des
Saugvermögens angegeben.
Vier mit \enquote{Leck} gekennzeichnete Punkte sind die aus den jeweiligen
Leckratenmessungen bestimmten Saugvermögen. Dabei ist die Unsicherheit des
Druckes die Unsicherheit des jeweilig verwendeten Druckmessgerätes.
Schließlich sind die mit \enquote{Evak} gekennzeichneten Punkte die aus der
jeweiligen Evakuierungsmessung bestimmten Saugvermögen. Hier beschreibt die
Unsicherheit auf der Druckachse den Druckbereich, auf welchem die Regression
durchgeführt wurde.
Sowohl bei der Evakuierungsmessung, als auch bei der Leckratenmessung wurden die
Regressionen vom kleineren Druck (Gleichgewichtsdruck beziehungsweise Druckbereich)
zum größeren Druck durchnummeriert.

\begin{figure}
  \centering
  \includegraphics[height=8cm]{build/vergleich_D.pdf}
  \caption{Saugvermögen der Drehschieberpumpe.}
  \label{fig:vgl_D}
\end{figure}
\begin{figure}
  \centering
  \includegraphics[height=8cm]{build/vergleich_T.pdf}
  \caption{Saugvermögen der Turbomolekularpumpe.}
  \label{fig:vgl_T}
\end{figure}
\FloatBarrier

% Erklärung Abweichung der Saugvermögen
Bei der Drehschieberpumpe (Abbildung \ref{fig:vgl_D}) zeigt sich,
dass die zwei Leckratenmessungen mit den
höchsten beiden Gleichgewichtsdrücken sowie die Evakuierungsregression im
mittleren Druckbereich unter Berücksichtigung ihres Fehlers mit der Herstellerangabe
überein stimmen. Die restlichen bestimmten Saugvermögen liegen unterhalb der
Herstellerangabe, was systematische Fehler nahe legt.
Anders sehen die Ergebnisse bei der Turbomolekularpumpe
(Abbildung \ref{fig:vgl_T}) aus. Hier ist eine deutliche Abweichung der
Saugvermögen nach unten von mindestens \SI{52}{\liter\raiseto{-1}\second}
feststellbar, sodass hier signifikante systematische Fehler vorliegen sollten.
% TODO Welche?

% Erklärung der Abweichung zu den Herstellerangaben
Ein Teil der Abweichungen können mit dem oben genannten Unsicherheiten in
der Zeitmessung erklärt werden. Des Weiteren sind die Herstellerangaben
unter idealen Bedingungen und ausschließlich mit Stickstoffgas gemessen
worden. Weiterhin wurde bei den Messungen des Herstellers der größte
Wert des Saugvermögens angegeben. Deshalb ist ein zu klein bestimmtes
Saugvermögen im Vergleich zur Herstellerangabe realistisch. Bei der
Durchführung wurde Luft abgepumpt, welches jedoch zum größten Teil aus
Stickstoff besteht, jedoch ein Stoffgemisch ist, sodass hier gerade bei
den niedrigeren Drücken der Turbomolekularpumpe eine Abweichung nach unten
zu erwarten ist.
% TODO Warum nach unten?

Des Weiteren wurde angenommen, dass alle Messungen bei einer konstanten
Temperatur durchgeführt wurden. Dies konnte jedoch nicht sicher gestellt werden.
In der Herleitung des Saugvermögens ist auch angenommen worden, dass das
Saugvermögen unabhängig vom Druck ist. Dies kann nach den Abbildungen
\ref{fig:vgl_D} und \ref{fig:vgl_T} ebenfalls nicht bestätigt werden.

Ein weiterer großer systematischer Fehler ist der nicht verschwindende Leitwert
des Rezipienten, welche zum Beispiel durch Schläuche mit geringem Durchmesser
in einem niedrigeren Saugvermögen resultiert (vergleiche Gleichung \eqref{eqn:effSaug}).
% TODO Genauer und Ausführlicher
% Qualitative Betrachtung, wie ändert sich z.B. das Saugvermögen bei
% Halbierung des Querschnitts (bei der Turbopumpe, molekulare Strömung)

Schließlich wurden die Ventile per Hand geschlossen, sodass nicht garantiert
werden kann, dass diese absolut dicht und insbesondere bei der
Messung der Evakuierungskurve beim Start der Messung bereits geschlossen waren.
Sollte das Schließen zu lange gedauert haben, so sinkt der Druck aufgrund der
einströmenden Luft langsamer und die bestimmte Steigung der Regression an die
logarithmierten Drücke ist kleiner.
Daraus resultiert dann ein kleineres Saugvermögen. Dies erklärt, warum die
aus der Evakuierungsmessung bestimmten Saugvermögen tendenziell kleiner sind
als die Saugvermögen der Leckratenmessungen.

Als Letztes sei erwähnt, dass trotz eines Ausheizens vor Beginn der Messungen
Desorption statt gefunden haben kann, was wiederum zu kleineren Saugvermögen
als die Herstellerangabe führt. Desorption tritt stärker auf, je kleiner der
Druck ist, weshalb dieser Effekt vor allem bei der Turbomolekularpumpe einen
systematischen Fehler ausmacht.
