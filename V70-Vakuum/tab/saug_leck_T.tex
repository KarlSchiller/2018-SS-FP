\begin{table}
    \centering
    \caption{Steigungen und Saugvermögen der Regressionen an die Leckratenmessungen der Turbomolekularpumpe.}
    \label{tab:saug_leck_T}
    \sisetup{table-format=2.1}
    \begin{tabular}{
                    S[table-format=2.0, fixed-exponent=-5, table-omit-exponent]
                    @{${}\pm{}$}
                    S[table-format=1.1, fixed-exponent=-5, table-omit-exponent]
                    S[table-format=2.2, fixed-exponent=-5, table-omit-exponent]
                    @{${}\pm{}$}
                    S[table-format=1.2, fixed-exponent=-5, table-omit-exponent]
                    S @{${}\pm{}$} S[table-format=1.1]}
    \toprule
        \multicolumn{2}{c}{$p_\text{G}\:/\:10^{-5}\;\si{\milli\bar}$} &
        \multicolumn{2}{c}{$m\:/\:10^{-5}\;\si{\milli\bar\raiseto{-1}\second}$} &
        \multicolumn{2}{c}{$S\:/\:\si{\liter\raiseto{-1}\second}$} \\
    \midrule
    % p_G und m werden durch 1e-5 dividiert
    5e-05 & 0.5e-05 & 6.70e-05 & 1.5e-06 & 14.6 & 1.8 \\
    1e-04 & 1e-05 & 21.59e-05 & 4.9e-06 & 23.6 & 3.0 \\
    15e-05 & 15e-06 & 32.49e-05 & 7.2e-06 & 23.7 & 3.0 \\
    20e-05 & 2e-05 & 45.81e-05 & 1.10e-05 & 25.0 & 3.2 \\
    \end{tabular}
\end{table}
