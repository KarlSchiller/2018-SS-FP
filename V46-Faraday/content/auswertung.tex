\newpage
\section{Auswertung}
\label{sec:Auswertung}

Im Folgenden werden die Regressionen, Abbildungen und Fehlerrechnungen mit den
Paketen Numpy \cite{numpy}, Uncertainties \cite{uncertainties}, Matplotlib
\cite{matplotlib} und Scipy \cite{scipy} in der Programmierumgebung
Python 3.6.5 erstellt beziehungsweise durchgeführt.

\subsection{Messung der magnetischen Flussdichte}
\label{sec:flussmessung}

Wie in Abschnitt \ref{sec:MessungBFeld} beschrieben wird das Magnetfeld
vor und nach der Messung der Faraday-Rotationswinkel vermessen.
Da jedoch vor Messung der Rotationswinkel der Elektromagnet noch nicht
aufgeheizt war, ist für die weitere Auswertung die Messung des Magnetfeldes
nach Messung der Rotationswinkel verwendet worden.
Die aufgenommenen Flussdichten $B$ an der jeweiligen Position $z$
sind in Tabelle \ref{tab:flussmessung} dargestellt.
Dabei entspricht die Reihenfolge der Messwerte der Reihenfolge ihrer
Aufnahme, die per Auge abschätzte Position der Probe $z = 0$ wurde
mehrmals vermessen.

\begin{table}
  \centering
  \caption{Die gemessenen magnetischen Flussdichten und ihre jeweilige Position.}
  \label{tab:flussmessung}
  \sisetup{table-format=3.0}
  \begin{tabular}{S[table-format=+2.0] S | S[table-format=+2.0] S}
  \toprule
    {$z\:/\:\si{\milli\meter}$} & {$B\:/\:\si{\milli\tesla}$} &
    {$z\:/\:\si{\milli\meter}$} & {$B\:/\:\si{\milli\tesla}$} \\
  \midrule
  0 & 376 & 0 & 375 \\
  -1 & 377 & 1 & 373 \\
  -2 & 376 & 2 & 369 \\
  -3 & 375 & 3 & 363 \\
  -4 & 372 & 4 & 354 \\
  -5 & 368 & 5 & 341 \\
  -6 & 362 & 6 & 325 \\
  -7 & 353 & 7 & 301 \\
  -8 & 340 & 8 & 268 \\
  -9 & 322 & 9 & 225 \\
  -10 & 294 & 10 & 175\\
   & & 0 & 374 \\
%  0 & 375 \\
%  1 & 373 \\
%  2 & 369 \\
%  3 & 363 \\
%  4 & 354 \\
%  5 & 341 \\
%  6 & 325 \\
%  7 & 301 \\
%  8 & 268 \\
%  9 & 225 \\
%  10 & 175 \\
%  0 & 374 \\
\end{tabular}
\end{table}


Zur Bestimmung der maximalen Flussdichte wird ein Polynom vierten Grades
der Form
\begin{equation}
  B_\text{reg} = a z^4 + b z^3 + c z^2 + d z + e
  \label{eqn:feldfit}
\end{equation}
mit Hilfe der Funktion \texttt{scipy.curve\_fit} an die Messwerte regressiert.
Diese Regression ist in blau zusammen mit den Messwerten in Abbildung
\ref{fig:flussmessung} dargestellt. Für die Parameter ergeben sich die Werte
\begin{align*}
  a &= \SI{-0.0081(2)}{\milli\tesla\per\raiseto{4}\milli\meter} \\
  b &= \SI{-0.045(1)}{\milli\tesla\per\raiseto{3}\milli\meter} \\
  c &= \SI{-0.59(2)}{\milli\tesla\per\raiseto{2}\milli\meter} \\
  d &= \SI{-1.58(8)}{\milli\tesla\per\raiseto{1}\milli\meter} \\
  e &= \SI{375.0(3)}{\milli\tesla}.
\end{align*}
Als die an der Probe anliegende Feldstärke wird das Maximum der magnetischen
Flussdichte angenommen. Dieses ergab sich unter Verwendung des Pakets
\texttt{scipy.minimize\_scalar} zu
\SI{376}{\milli\tesla} an der Position $z = \SI{-1.5}{\milli\meter}$.

\begin{figure}
  \centering
  \includegraphics[height=8cm]{build/feldmessung2.pdf}
  \caption{Die gemessene magnetische Flussdichte in Abhängigkeit des Ortes.}
  \label{fig:flussmessung}
\end{figure}
\FloatBarrier


\subsection{Bestimmung der effektiven Masse}
\label{sec:AuswMasse}

Im Folgenden werden die Proben nach der in Abschnitt \ref{sec:MessungFaraday}
angegebenen Reihenfolge durchnummeriert.
Für jede Probe wird pro Filterwellenlänge die Differenz der am Goniometer
abgelesenen Winkel vor und nach Umpolung des Magnetfelds gebildet.
Diese Differenzwinkel $\theta$ sind zusammen mit der jeweiligen Wellenlänge
$\lambda$ in den Tabellen \ref{tab:probe1} bis \ref{tab:probe3}
aufgeführt.
Um nun die verschiedenen Proben miteinander vergleichen zu können,
werden die Rotationswinkel anschließend in das Radmaß umgerechnet und
auf die Dicke der Probe $L$ normiert.
Die entsprechenden Werte sind ebenfalls in den drei Tabellen
eingetragen und grafisch in Abbildung \ref{fig:winkel}
gegen das Quadrat der Wellenlänge dargestellt.

\begin{table}
  \centering
  \caption{Messwerte und weitere Größen zu Probe 1.}
  \label{tab:probe1}
  \sisetup{table-format=3.2}
  \begin{tabular}{S[table-format=1.2] S[table-format=1.2]
                  S[table-format=2.2] S S}
  \toprule
    {$\lambda\:/\:\si{\micro\meter}$} &
    {$\lambda^2\:/\:(\si{\micro\meter})^2$} &
    {$\theta\:/\:\si{\degree}$} &
    {$\frac{\theta}{L}\:/\:\si{\radian\raiseto{-1}\meter}$} &
    {$\increment\frac{\theta}{L}\:/\:\si{\radian\raiseto{-1}\meter}$} \\
  \midrule
  1.6 & 2.56 & 20.50 & 138.04 & 103.03 \\
  1.55 & 2.40 & 18.07 & 121.65 & 90.80 \\
  1.5 & 2.25 & 19.13 & 128.83 & 96.16 \\
  1.4 & 1.96 & 19.97 & 134.45 & 100.35 \\
  1.3 & 1.69 & 14.97 & 100.78 & 75.22 \\
  1.2 & 1.44 & 21.23 & 142.98 & 106.71 \\
  1.1 & 1.21 & 19.17 & 129.06 & 96.33 \\
  1.0 & 1.00 & 38.50 & 259.24 & 193.49 \\
\end{tabular}
\end{table}

\begin{table}
  \centering
  \caption{Messwerte und weitere Größen zu Probe 2.}
  \label{tab:probe2}
  \sisetup{table-format=3.2}
  \begin{tabular}{S[table-format=1.2] S[table-format=1.2]
                  S[table-format=2.2] S S}
  \toprule
    {$\lambda\:/\:\si{\micro\meter}$} &
    {$\lambda^2\:/\:(\si{\micro\meter})^2$} &
    {$\theta\:/\:\si{\degree}$} &
    {$\frac{\theta}{L}\:/\:\si{\radian\raiseto{-1}\meter}$} &
    {$\increment\frac{\theta}{L}\:/\:\si{\radian\raiseto{-1}\meter}$} \\
  \midrule
  1.6 & 2.56 & 20.50 & 131.54 & 96.53 \\
  1.55 & 2.40 & 18.07 & 115.93 & 85.07 \\
  1.5 & 2.25 & 19.13 & 122.77 & 90.10 \\
  1.4 & 1.96 & 19.97 & 128.12 & 94.02 \\
  1.3 & 1.69 & 14.97 & 96.04 & 70.48 \\
  1.2 & 1.44 & 21.23 & 136.25 & 99.99 \\
  1.1 & 1.21 & 19.17 & 122.99 & 90.25 \\
  1.0 & 1.0 & 38.50 & 247.04 & 181.29 \\
\end{tabular}
\end{table}

\begin{table}
  \centering
  \caption{Messwerte und weitere Größen zu Probe 3.}
  \label{tab:probe3}
  \sisetup{table-format=2.2}
  \begin{tabular}{S[table-format=1.2] S[table-format=1.2] S S}
  \toprule
    {$\lambda\:/\:\si{\micro\meter}$} &
    {$\lambda^2\:/\:(\si{\micro\meter})^2$} &
    {$\theta\:/\:\si{\degree}$} &
    {$\frac{\theta}{L}\:/\:\si{\radian\raiseto{-1}\meter}$} \\
  \midrule
  1.6 & 2.56 & 20.50 & 35.01 \\
  1.55 & 2.40 & 18.07 & 30.85 \\
  1.5 & 2.25 & 19.13 & 32.68 \\
  1.4 & 1.96 & 19.97 & 34.10 \\
  1.3 & 1.69 & 14.97 & 25.56 \\
  1.2 & 1.44 & 21.23 & 36.26 \\
  1.1 & 1.21 & 19.17 & 32.73 \\
  1.0 & 1.0 & 38.50 & 65.75 \\
\end{tabular}
\end{table}

\begin{figure}
  \centering
  \includegraphics[height=8cm]{build/winkel.pdf}
  \caption{Differenz der Rotationswinkel zwischen den Magnetfeldpolungen für
  alle drei Proben.}
  \label{fig:winkel}
\end{figure}
\FloatBarrier

Um die Faraday-Rotation von freien Ladungsträgern zu untersuchen,
wurde der Rotationswinkel der hochreinen GaAs-Probe (Probe 3)
von den ersten beiden Proben abgezogen. Diese Differenzen sind
in Tabelle \ref{tab:probe1} bis \ref{tab:probe3}
als $\increment\sfrac{\theta}{L}$ aufgeführt.
Nach Formel \eqref{eqn:theta-frei} ist der auf die Länge normierte
Rotationswinkel proportional zum Quadrat der Filterwellenlänge.
Aufgrund dessen wird eine Funktion der Form
\begin{equation*}
  \frac{\theta}{L}\left(\lambda\right) = a \cdot \lambda^2 + b
\end{equation*}
an die Messwerte regressiert.
Unter Verwendung von Formel \eqref{eqn:theta-frei} lässt sich daraus die
effektive Masse der freien Ladungsträger in GaAs zu
\begin{equation}
  m^* = \sqrt{\frac{e^3}{8\:\pi\:\epsilon_0\:c^3}
  \frac{1}{a}\frac{N\:B}{n}}
  \label{eqn:mstar}
\end{equation}
berechnen.
Als Brechungsindex wurde \num{3.397} eingesetzt, welcher für eine
Wellenlänge von \SI{1377.56}{\nano\meter} in Quelle \cite{filmetrics} angegeben ist.
Für die beiden dotierten Proben 1 und 2 ergaben sich unter Verwendung
der Funktion \texttt{scipy.curve\_fit} die Regressionsparameter zu
\begin{align*}
  a_1 &= \SI{03(13)e12}{\radian\per\raiseto{3}\meter} \\
  b_1 &= \SI{90(27)e12}{\radian\per\meter} \\
  a_2 &= \SI{02(12)e12}{\radian\per\raiseto{3}\meter} \\
  b_2 &= \SI{85(25)e12}{\radian\per\meter}.
\end{align*}
Bei der Regression wurden die Messwerte für eine Wellenlänge von
\SI{1}{\micro\meter} nicht mit einbezogen.
In Abbildung \ref{fig:differenzen} sind die Differenzwinkel mit den
Regressionen dargestellt.
Unter Verwendung von Gleichung \eqref{eqn:mstar} ergeben sich die
effektiven Massen zu
\begin{align*}
  \frac{m^*_1}{m_\text{e}} &= \num{0.011(27)} \text{\quad und} \\
  \frac{m^*_2}{m_\text{e}} &= \num{0.007(19)}.
\end{align*}

\begin{figure}
  \centering
  \includegraphics[height=8cm]{build/differenzen.pdf}
  \caption{Rotationswinkel für freie Ladungsträger mit Regression für die beiden
  dotierte Proben.}
  \label{fig:differenzen}
\end{figure}
