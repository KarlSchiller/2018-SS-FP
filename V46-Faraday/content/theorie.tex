\section{Zielsetzung}
\label{sec:Zielsetzung}

In diesem Versuch wird die effektive Masse von Leitungselektronen in
GaAs ermittelt.
Dazu wird die Faraday-Rotation an unterschiedlich stark
dotierte Proben für verschiedene Wellen gemessen.

\section{Theorie}
\label{sec:Theorie}

\subsection{Die effektive Masse}
\label{sec:effMasse}

Halbleiter haben im Allgemeinen eine komplizierte Bandstruktur.
Zur einfacheren Handhabung kann in der Nähe des Minimums des
Leitungsbands am Punkt $\vec{k} = \vec{0}$ die $E\left(\vec{k}\right)$-
Relation in eine Taylorreihe zu
\begin{equation*}
  E\left(\vec{k}\right) = E_\text{0} +
  \frac{1}{2} \sum_{i = 1}^3
  \left(\frac{\partial^2 E}{\partial k_\text{i}^2}\right)_{\vec{k} = \vec{0}}
  k_\text{i} + ...
\end{equation*}
entwickelt werden.
Dabei steht $\vec{k}$ für den Wellenvektor des Elektrons und der Index
$i$ für die Raumdimensionen.
Diese Reihe lässt sich in die parabolische Relation eines freien Elektrons
überführen, wobei die Größe
\begin{equation}
  m_\text{i}^* := \hbar^2
  \left(\frac{\partial^2 E}{\partial k_\text{i}^2}\right)_{\vec{k} = \vec{0}}^{-1}
  \label{eqn:effMasse}
\end{equation}
als \emph{effektive Masse} definiert wird,
da sie die Dimension einer Masse haben muss.
In der Formel spielt auch das reduzierte Plancksche
Wirkungsquantum $\hbar$ eine Rolle.
Bei einer Substitution der Elektronenmasse durch die effektive Masse lässt
sich somit für Leitungselektronen die Quantenmechanik freier Teilchen
anwenden, der Einfluss des Kristallpotentials ist durch dieses
Konzept berücksichtigt.

\subsection{Rotation linear polarisierten Lichts in doppelbrechenden Kristallen}
\label{sec:Theo1}

Linear polarisiertes Licht lässt sich als eine lineare Überlagerung von
links- und rechtszirkular polarisierten Lichts auffassen.
Durchläuft dieses einen sogenannten doppelbrechenden Kristall der Länge $L$,
so haben die beiden zirkularen Wellen unterschiedliche Phasengeschwindigkeiten
und infolge dessen wird die lineare (überlagerte) Polarisationsrichtung
um einen Winkel von
\begin{equation}
  \theta = \frac{L}{2} \left(\vec{k_\text{R}} - \vec{k_\text{L}}\right)
  \label{eqn:Winkel-2}
\end{equation}
gedreht.
Dabei steht $\vec{k_\text{R}}$ für den Wellenvektor des rechtszirkular und
$\vec{k_\text{L}}$ für den Wellenvektor des linkszirkular polarisierten Lichts.
Dieser Effekt lässt sich mittels elektrischer Dipole erlären,
welche im Kristall induziert werden:
In einem externen elektrischen Feld $\vec{E}$ werden zum einen positive Ionen relativ
zu negativen Ionen verschoben, oder zum anderen der Ladungsschwerpunkt der
positiven Ladungen und der Elektronenverteilung stimmt nicht überein.
Die Summe der einzelnen Dipolmomente pro Volumen erzeugt eine Polarisation
\begin{equation}
  \vec{P} = \epsilon_0 \; \chi \; \vec{E}
  \label{eqn:Polarisation}
\end{equation}
mit der elektrischen Feldkonstanten $\epsilon_0$ und der
\emph{dielektrischen Suszeptibilität} $\chi$.
Dabei ist die Suszeptibilität $\chi$ im Allgemeinen ein Tensor.

Ein Material ist doppelbrechend, wenn im Suszeptibilitätstensor
nicht-diagonale und komplex konjugierte Koeffizienten auftreten.
Für in $z$-Richtung durchlaufendes Licht hat ein solcher Tensor
mit allen für die Drehung relevanten Koeffizienten die Gestalt
\begin{equation*}
  \chi =
  \begin{pmatrix}
    \chi_\text{xx} & i \chi_\text{xy} & 0 \\
    -i \chi_\text{xy} & \chi_\text{xx} & 0 \\
    0 & 0 & \chi_\text{zz} \\
  \end{pmatrix}
\end{equation*}
Ein Ausdruck für den Drehwinkel kann nun durch Einsetzen des Ansatzes
einer ebenen Welle in die Wellengleichung in Materie gewonnen werden.
Es ergeben sich nach Zerlegung in die Komponenten drei Gleichungen,
aus einer folgt $E_\text{z} = 0$.
% Es liegt also eine transversale Wellenausbreitung vor.
Für die beiden übrigen Gleichungen muss für eine nicht-trivial Lösung
\begin{equation*}
  \left(- k^2 + \frac{\omega^2}{c^2} \left(1 + \chi_\text{xy}\right)\right)^2
  -\left(i \frac{\omega^2}{c^2} \chi_\text{xy}\right)
  \left(- i \frac{\omega^2}{c^2} \chi_\text{xy}\right)
  = 0
\end{equation*}
mit der Lichtgeschwindigkeit $c$ und der Kreisfrequenz $\omega$
der Welle gelten.
Daraus folgen zwei mögliche Wellenzahlen
\begin{equation*}
  k_\pm = \frac{\omega}{c} \sqrt{\left(1 + \chi_\text{xx}\right) \pm \chi_\text{xy}},
\end{equation*}
oder mit der Beziehung $v_\text{Ph} = \omega / k$ zwei Phasengeschwindigkeiten
% \begin{align*}
%   v_\text{Ph,R} &=
%   \frac{c}{\sqrt{1 + \chi_\text{xx} + \chi_\text{xy}}}\\
%   v_\text{Ph,L} &=
%   \frac{c}{\sqrt{1 + \chi_\text{xx} - \chi_\text{xy}}}\\
% \end{align*}
für eine rechts- beziehungsweise linkszirkular polarisierte Welle.
Werden diese Wellenzahlen in den Ausdruck \eqref{eqn:Winkel-2}
eingesetzt und dieser linearisiert, ergibt sich der Ausdruck
\begin{equation}
  \theta \approx \frac{L \, \omega}{2 \, c \, n} \chi_\text{xy}
  \label{eqn:Winkel-1}
\end{equation}
mit dem Brechungsindex $n = c / v_\text{Ph}$.


\subsection{Faraday-Rotation in optisch inaktiven Medien}
\label{sec:Theo22}

In einem optisch inaktiven Medium dreht sich die Polarisationsrichtung von
durchgehendem Licht nicht ($\chi_\text{ik} = 0$ für $i \neq k$).
Dies ändert sich jedoch bei Anwesenheit eines externen magnetischen Feldes $\vec{B}$
parallel zur Ausbreitungsrichtung des Lichts.
Ein Elektron wird mit der Gleichung
\begin{equation*}
  m_\text{e} \frac{\symup{d}^2 \vec{r}}{\symup{d} t^2} + K \, \vec{r} =
  - e_0 \vec{E}\left(r\right)
  - e_0 \frac{\symup{d} \vec{r}}{\symup{d} t} \times \vec{B}
\end{equation*}
beschrieben, wobei $m_\text{e}$ die Masse und $e_0$ die Ladung des Elektrons,
$K$ eine die Bindung beschreibende Konstante und
$\vec{E}$ die Feldstärke der einfallenden Lichtwelle darstellt.
Wird des Weiteren angenommen, dass $\vec{E}$ in der Zeit harmonisch
oszilliert, die Polarisation $\vec{P}$ proportional zu $\vec{r}$ ist
und das Magnetfeld in z-Richtung zeigt, lässt sich diese Gleichung
komponentenweise zerlegen.
Nicht-triviale Lösungen ergeben sich nur für $E_\text{z} = 0$,
damit reduziert sich die Anzahl an Komponentengleichungen auf zwei.
Wird die Relation \eqref{eqn:Polarisation} mit dem Ansatz
\begin{equation*}
  \chi =
  \begin{pmatrix}
    \chi_\text{xx} & i \chi_\text{xy} & 0 \\
    i \chi_\text{yx} & \chi_\text{xx} & 0 \\
    0 & 0 & \chi_\text{zz} \\
  \end{pmatrix}
\end{equation*}
in die verbleibenden zwei Komponenten eingesetzt und diese Gleichung
nach Real- und Imaginärteil getrennt, so ergibt sich durch Vergleich
\begin{equation*}
  \chi_\text{xy} = - \chi_\text{yx}.
\end{equation*}
Somit dreht sich die Polarisationsebene von einfallendem Licht in einem
optisch inaktiven Medium bei einem externen Magnetfeld parallel zur
Ausbreitungsrichtung prinzipiell.
Dieser Effekt wird \emph{Faraday-Effekt} genannt.

Mit Gleichung \eqref{eqn:Winkel-1} und den eben genannten Überlegungen lässt
sich der Drehwinkel durch
\begin{equation*}
  \theta = \frac{e_0^3}{2 \epsilon_0 c}\frac{1}{m_\text{e}^2}
  \frac{\omega^2}{\left(-\omega^2 + \frac{K}{m_\text{e}}\right)^2 -
  \left(\frac{e_0}{m_\text{e}} B \omega\right)^2}
  \frac{N B L}{n}
\end{equation*}
mit der Masse des Elektrons $m_\text{e}$ und der Anzahl an Elektronen pro
Volumen $N$ ausdrücken.
In diesem Ausdruck steht zum einen die
Zyklotronfrequenz $\omega_\text{Z} = B e_0 / m_\text{e}$ (Frequenz eines im magnetischen
Feld rotierenden Elektrons aufgrund der Lorentzkraft) und zum anderen
der Ausdruck $\omega_0 = \sqrt{K / m_\text{e}}$,
welcher die Bedeutung einer Resonanzfrequenz hat.
Für \emph{quasifreie} Ladungsträger gilt der Grenzfall $\omega_0 \rightarrow 0$
und damit
\begin{equation}
  \theta_\text{frei} \approx \frac{e_0^3}{8 \pi^2 \epsilon_0 c^3}
  \frac{1}{m_\text{e}^2} \lambda^2 \frac{N B L}{n}.
  \label{eqn:theta-frei}
\end{equation}
Ansonsten ist in Halbleitern die Resonanzfrequenz $\omega_0$ in der Regel
deutlich größer als die Zyklotronfrequenz, sodass die Näherungen
\begin{equation*}
  \left(\omega_0^2 - \omega^2\right)^2 >> \omega^2 \omega_\text{Z}^2
  \text{    und    }
  \omega << \omega_0
\end{equation*}
angewendet werden können.
Hieraus ergibt sich die Relation
\begin{equation}
  \theta\left(\lambda\right) \approx \frac{2 \pi^2 e_0^3 c}{\epsilon_0}
  \frac{1}{m_\text{e}^2}\frac{1}{\lambda^2 \omega_0^4}
  \frac{N B L}{n}.
  \label{eqn:theta-gebunden}
\end{equation}
