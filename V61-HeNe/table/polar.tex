\begin{table}
\centering
\begin{tabular}{cccccc}
\toprule
$\phi$ / $\si{\degree}$ & $\phi$ / $\si{\radian}$ & $I$ / $\si{\nano\ampere}$ & $\phi$ / $\si{\degree}$ & $\phi$ / $\si{\radian}$ & $I$ / $\si{\nano\ampere}$ \\
\midrule
10.0 & 0.17 & 508.0 & 190.0 & 0.17 & 508.0 \\
20.0 & 0.35 & 910.0 & 200.0 & 0.35 & 910.0 \\
30.0 & 0.52 & 1370.0 & 210.0 & 0.52 & 1370.0 \\
40.0 & 0.7 & 1760.0 & 220.0 & 0.7 & 1760.0 \\
50.0 & 0.87 & 2680.0 & 230.0 & 0.87 & 2680.0 \\
60.0 & 1.05 & 2800.0 & 240.0 & 1.05 & 2800.0 \\
70.0 & 1.22 & 3300.0 & 250.0 & 1.22 & 3300.0 \\
80.0 & 1.4 & 3180.0 & 260.0 & 1.4 & 3180.0 \\
90.0 & 1.57 & 3100.0 & 270.0 & 1.57 & 3100.0 \\
100.0 & 1.75 & 2420.0 & 280.0 & 1.75 & 2420.0 \\
110.0 & 1.92 & 1720.0 & 290.0 & 1.92 & 1720.0 \\
120.0 & 2.09 & 1220.0 & 300.0 & 2.09 & 1220.0 \\
130.0 & 2.27 & 780.0 & 310.0 & 2.27 & 780.0 \\
140.0 & 2.44 & 376.0 & 320.0 & 2.44 & 376.0 \\
150.0 & 2.62 & 127.0 & 330.0 & 2.62 & 127.0 \\
160.0 & 2.79 & 11.2 & 340.0 & 2.79 & 11.2 \\
170.0 & 2.97 & 29.7 & 350.0 & 2.97 & 29.7 \\
\bottomrule
\end{tabular}
\caption{Aufgenommene Messwerte zur Untersuchung der Polarisation des Lichtstrahls.
Aufgelistet sind der Winkel $\phi$, in Grad, sowie in Radiant. Zudem ist die bei dem
Winkel auftretende Lichtintensität nachzulesen.}
\label{tab:pol}
\end{table}
