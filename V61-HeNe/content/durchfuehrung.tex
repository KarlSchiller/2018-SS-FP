\newpage
\section{Durchführung}
\label{sec:Durchführung}
Während dieses Experimentes befinden sich alle Bestandteile auf einer Schiene,
damit die relativen Abstände variiert werden können. Der gesammte Aufbau ist in
Abbilsung \ref{fig:Versuchsaufbau1} abgebildet.
\begin{figure}[htb]
  \centering
  \includegraphics[width=0.8\textwidth]{content/aufbau.png}
  \caption{Abbildung des Aufbaus eines HeNe-Lasers mit zur Justage verwendetem JUstagelaser.}
  \label{fig:Versuchsaufbau1}
\end{figure}
\FloatBarrier

\subsection{Justage}

Damit der Versuch durchgeführt
werden kann, müssen die einzelnen Bestandteile zuerst justiert werden. Zu diesem
Zweck wird ein weiterer Laser als Justage-Laser und zwei Lochblenden verwendet.
Um die notwendige Gasentladung zu erhalten, wird eine Hochspannung an das Laserrohr
angelegt. Um nun den Laser zum lasern zu bringen werden die konfokalen
Resonatorspiegel mit Justierschrauben  an Spiegeln und laserrohr so eingestellt,
sodass alle optischen Achsen aufeinander liegen. Dabei ist eine Photodiode zu
Intensitätsmessung hinter dem Laser angebracht.

\subsection{Bestimmung der Wellenlänge}
Um die Wellenlänge
des erzeugten Lasers zu messen wird ein optisches Gitter verwendet.
Um das entstehende Interferenzbild zu vermessen wird hinter das Gitter ein Schirm
gestellt, an dem die Abstände zwischen den auftretenden Interferrenzmaxima vermessen
werden können. Daraus, und aus der Messung des Abstandes zwischen Schirm und Gitter
kann die Wellenlänge durch
\begin{equation}
  \lambda = \frac{g\cdot\sin(\phi)}{n},\ \ \phi = \arctan\left(\frac{d_n}{L}\right),\  n\in\mathds{N}
\end{equation}
berechnet werden. Hierbei ist $g$ die Gitterkonstante, $d_n$ der Abstand zwischen Hauptmaxima
und $n$-ten Maximum und $L$ der Abstand zwischen Gitter und Schirm.

\subsection{Unteruchung der TEM-Moden}
Um die TEM-Moden untersuchen zu können wird eine defokussierende Linse hinter den Laser gestellt.
So wird der Strahl des Lasers verbreitert und die Untersuchung erleichtert werden.
Um die Photodiode senkrecht zu der Starhlenachse verschieben zu können werden
Mikrometerschraube verwendet. Durch Verschieben der Photodiode kann die Abhängigkeit
zwischen Intensität und Achsenabstand bestimmt werden.

Die Grundmode ist ohne Einsatz von Blenden und Gittern zu untersuchen. Die
$I_{01}$-Mode wird vermessen, indem ein Wolframdraht so positioniert wird, dass
die Grundmode unterdrückt wird. Die $I_{01}$-Mode besitzt eine Nullstelle bei $r = 0$
und wird ebenfalls durch verschieben der Photodiode senkrecht zur Strahlenachse
untersucht.

\subsection{Untersuchung der Polarisation}
Um zu untersuchen welche Polarisation der Laser besitzt, sind an den Ausgängen des
Laserrohrs Brewster-Fenster angebracht. Brewster-Fenster sind gläsernde Platten, die einen
zur optischen Achse eingestellten Brewsterwinkel besitzen. Als Brewsterwinkel wird
jener Winkel bezeichnet, bei dem nach den Fresnelschen Formel kein parallel zur
Einfallsebene polarisertes Licht reflektiert wird. Zudem wird das senkrecht zur
Einfallsebene polarisierte Licht durch Reflexion stark unterdrückt. Somit ist der
verbleibende Lichtstrahl linear polarisiert.

Zur Untersuchung dieser Begebenheit wird das Gesetz von Malus verwendet, welches
die Intensität des Strahls hinter dem Polasisationsfilter beschreibt. Mit
den konstanten Parametern $I_0$ und $\phi_0$ wird in Abhängigkeit des Drehwinkels
$\phi$ die verbleibende Intensität berechnet durch
\begin{equation}
  I(\phi) = I_0\sin^2(\phi+\phi_0).
\end{equation}

\subsection{Überprüfung der Stabilitätsbedingung}
Damit die Rinchtigkeit der Stabilitätsbedingung überprüft werden kann, wird die
variierte Resonatorlänge $L$ gegen den Photostrom aufgezeichnet werden. Hier
wird ein gekrümmter Spiegel $r=\SI{1.4}{\metre}$ und eine ebener Spiegel verwendet.
Bei jeder Variation der Resonatorlänge ist dabei die Justage neu durchzuführen,
um den Photostrom zu maximieren.
