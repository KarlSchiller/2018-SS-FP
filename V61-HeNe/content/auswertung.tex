\newpage
\section{Auswertung}

\label{sec:Auswertung}
Im Folgenden wurden alle Ausgleichsrechnungen mit der Funktion $curve\_fit$ von dem
Python Paket $scipy.optimize$ \cite{scipy} durchgeführt, sowie deren Parameter berechnet.

\subsection{Bestimmung der Wellenlänge}
Die zur Betsimmung der Wellenlängedes lasers aufgenommenen Messwerte
sind in Tabelle \ref{tab:wellenlänge}. Dabei wurde der Laser in einem Abstand
von $L = \SI{11.3(2)}{\centi\meter}$ positioniert und ein Gitter mit einem Abstand der einzelnen
Streben von $a = \SI{0.1}{\milli\meter}$ verwendet.

\begin{table}
\centering
\begin{tabular}{ccc}
\toprule
Ordnung $n$ & Abstand $d$ / $\si{meter}$ & Wellenlänge $\lambda$ / $\si{\nano\meter}$ \\
\midrule
1.0 & 0.068 & 515.61 \\
2.0 & 0.145 & 438.92 \\
3.0 & 0.214 & 353.82 \\
4.0 & 0.29 & 295.34 \\
5.0 & 0.376 & 252.99 \\
-1.0 & 0.079 & 572.98 \\
-2.0 & 0.14 & 431.23 \\
-3.0 & 0.212 & 352.61 \\
-4.0 & 0.288 & 294.78 \\
-5.0 & 0.367 & 251.69 \\
\bottomrule
\end{tabular}
\caption{Aufgenommene Messwerte zur Bestimmung der Wellenlänge des verwendeten Lasers.}
\label{tab:welle}
\end{table}


Für die Berechnung der Wellenlänge wurde Formel \ref{eqn:welle} verwendet. Dabei
beschreibt die Ordnung $n>0$ die auftretenden Maxima nach rechts und $n<0$ die
Maxima nach links.
Eine Mittelung über alle berechneten Wellenlängen ergab
\begin{align*}
  \bar{\lambda} = \SI{375.99(10489)}{\nano\meter}.
\end{align*}


\subsection{Untersuchung der TEM-Moden}

\subsubsection{Auswertung der $T_{00}$ Mode}
Die in Abbildung \ref{plt:t00} dargestellten Messwerte für die Intenstät $I$
und die Entfernung vom Hauptmaxima $r$ sind in Tabelle \ref{tab:t00}
aufgelistet.

\begin{table}
\centering
\caption{Aufgenommene Messwerte zur Untersuchung der $T_{00}$ Mode. Dabei ist der
Abstand zum Hauptmaxima, sowie die bei dem Anstand vermessene Lichtntensität aufgelistet.}
\begin{tabular}{cccc}
\toprule
Abstand $d$ / $\si{\centi\meter}$ & Intensität $I$ / $\si{\nano\ampere}$  & Abstand $d$ / $\si{\centi\meter}$ & Intensität $I$ / $\si{\nano\ampere}$ \\
\midrule
-28.0 & 5.3 & 2.0 & 3290.0 \\
-26.0 & 3.3 & 4.0 & 3000.0 \\
-24.0 & 12.9 & 6.0 & 2800.0 \\
-22.0 & 26.0 & 8.0 & 2150.0 \\
-20.0 & 32.0 & 10.0 & 1610.0 \\
-18.0 & 73.0 & 12.0 & 1150.0 \\
-16.0 & 148.0 & 14.0 & 820.0 \\
-14.0 & 275.0 & 16.0 & 470.0 \\
-12.0 & 430.0 & 18.0 & 270.0 \\
-10.0 & 815.0 & 20.0 & 140.0 \\
-8.0 & 1090.0 & 22.0 & 40.0 \\
-6.0 & 1470.0 & 24.0 & 35.0 \\
-4.0 & 2180.0 & 26.0 & 11.0 \\
-2.0 & 2630.0 & 28.0 & 9.0 \\
\bottomrule
\end{tabular}
\label{tab:t00}
\end{table}

\FloatBarrier

\begin{figure}[htb]
  \centering
  \includegraphics[width=0.7\textwidth]{T00.pdf}
  \caption{Grphisch Darstellung der Werte aus Tabelle \ref{tab:t00} mit Ausgleichskurve der Form von Formel \ref{eqn:t00}.}
  \label{plt:t00}
\end{figure}
\FloatBarrier

Die durch die Ausgleichsrechung berechneten Parameter ergeben:
\begin{align*}
  I_0 &= \SI{2.09(8)}{\nano\ampere} \\
  d_0 &= \SI{3212(31)}{\milli\meter} \\
  \omega &= \SI{13.87(16)}{\raiseto{-1}\milli\meter}
\end{align*}

\subsubsection{Auswertung der $T_{10}$}
Die für die Untersuchung der $T_{10}$ Mode aufgenommen Intensitäten $I$,
sowie Abstände $r$ sind in Tabelle \ref{tab:t10} aufgelistet und in
Abbildung \ref{plt:t10} zusammen mit einer berechneten Ausgleichskurve
graphisch dargestellt.

\begin{table}
\begin{tabular}{cccc}
col0 & col1 & col2 & col3 \\
-28.0 & 1.5 & 2.0 & 2.5 \\
-26.0 & 5.5 & 4.0 & 42.0 \\
-24.0 & 9.5 & 6.0 & 113.0 \\
-22.0 & 14.0 & 8.0 & 180.0 \\
-20.0 & 25.0 & 10.0 & 260.0 \\
-18.0 & 66.0 & 12.0 & 310.0 \\
-16.0 & 115.0 & 14.0 & 310.0 \\
-14.0 & 160.0 & 16.0 & 240.0 \\
-12.0 & 190.0 & 18.0 & 200.0 \\
-10.0 & 230.0 & 20.0 & 125.0 \\
-8.0 & 200.0 & 22.0 & 62.0 \\
-6.0 & 190.0 & 24.0 & 28.0 \\
-4.0 & 150.0 & 26.0 & 18.0 \\
-2.0 & 70.0 & 28.0 & 14.0 \\
\end{tabular}
\end{table}

\FloatBarrier

\begin{figure}[htb]
  \centering
  \includegraphics[width=0.7\textwidth]{T10.pdf}
  \caption{Darstellung der in Tabelle \ref{tab:t10} aufgelisteten Messwerte und einer berechneten Ausgleichskurve.}
  \label{plt:t10}
\end{figure}
\FloatBarrier

Die aus der Ausgleichsrechnung nach Formel \ref{eqn:t10} bestimmten
Parameter sind:

\begin{align*}
  I_{0,1} &= \SI{227(8)}{\nano\ampere} \ \ \ \ \ \ \ \ \ \ \ \ \ I_{0,2} = \SI{315(8)}{\nano\ampere} \\
  d_{0,1} &= \SI{-9.66(21)}{\milli\meter} \ \ \ \ d_{0,2} = \SI{13.19(14)}{\milli\meter} \\
  \omega_1 &= \SI{10.2(4)}{\raiseto{-1}\milli\meter} \ \ \ \ \ \omega_2 = \SI{9.60(29)}{\raiseto{-1}\milli\meter}
\end{align*}

\subsection{Untersuchung der Polarisation}
Die Tabelle \ref{tab:pol} enthält alle zur Untersuchung der Polarisation
des Lasers verwendeten Messdaten. Dabei ist der Winkel sowohl in Grad als
auch in Rad angegeben.
Die durch die Ausgleichsrechung in Abbildung \ref{plt:pol} bestimmten
Parameter ergaben sich zu
\begin{align*}
  I_0 &= \SI{2.83(6)}{\micro\ampere} \\
  \phi_0 &= \SI{-87.72(2)}{\radian} \\
      &= \SI{-13.96(0)}{\degree}
\end{align*}

\begin{figure}[htb]
  \centering
  \includegraphics[width=0.7\textwidth]{build/polarisation.pdf}
  \caption{Graphische Abbildung der Messdaten aus Tabelle \ref{tab:pol} mit zugehöriger Ausgleichskurve.}
  \label{plt:pol}
\end{figure}
\FloatBarrier

\begin{table}
\centering
\caption{Aufgenommene Messwerte zur Untersuchung der Polarisation des Lichtstrahls.
Aufgelistet sind der Winkel $\phi$, in Grad, sowie in Radiant. Zudem ist die bei dem
Winkel auftretende Lichtintensität nachzulesen.}
\label{tab:pol}
\begin{tabular}{cccccc}
\toprule
$\phi$ / $\si{\degree}$ & $\phi$ / $\si{\radian}$ & $I$ / $\si{\nano\ampere}$ & $\phi$ / $\si{\degree}$ & $\phi$ / $\si{\radian}$ & $I$ / $\si{\nano\ampere}$ \\
\midrule
10.0 & 0.17 & 508.0 & 190.0 & 0.17 & 508.0 \\
20.0 & 0.35 & 910.0 & 200.0 & 0.35 & 910.0 \\
30.0 & 0.52 & 1370.0 & 210.0 & 0.52 & 1370.0 \\
40.0 & 0.7 & 1760.0 & 220.0 & 0.7 & 1760.0 \\
50.0 & 0.87 & 2680.0 & 230.0 & 0.87 & 2680.0 \\
60.0 & 1.05 & 2800.0 & 240.0 & 1.05 & 2800.0 \\
70.0 & 1.22 & 3300.0 & 250.0 & 1.22 & 3300.0 \\
80.0 & 1.4 & 3180.0 & 260.0 & 1.4 & 3180.0 \\
90.0 & 1.57 & 3100.0 & 270.0 & 1.57 & 3100.0 \\
100.0 & 1.75 & 2420.0 & 280.0 & 1.75 & 2420.0 \\
110.0 & 1.92 & 1720.0 & 290.0 & 1.92 & 1720.0 \\
120.0 & 2.09 & 1220.0 & 300.0 & 2.09 & 1220.0 \\
130.0 & 2.27 & 780.0 & 310.0 & 2.27 & 780.0 \\
140.0 & 2.44 & 376.0 & 320.0 & 2.44 & 376.0 \\
150.0 & 2.62 & 127.0 & 330.0 & 2.62 & 127.0 \\
160.0 & 2.79 & 11.2 & 340.0 & 2.79 & 11.2 \\
170.0 & 2.97 & 29.7 & 350.0 & 2.97 & 29.7 \\
\bottomrule
\end{tabular}
\end{table}

\FloatBarrier

\subsection{Überprüfung der Stabilitätsbedingung}
Bei der Untersuchung der Stabilitätsbedingung werden zwei verschiedene
Konfigurationen Untersucht. Dabei wird bei der ersten Untersuchung zwei
konfokale Spiegel verwendet, bei der zweiten hingegen ein konfokaler und
ein ebener Spiegel. Dabei wird eine Umskalierung der Messwerte der Art
\begin{align*}
  I \rightarrow \frac{I \cdot c}{I_\text{max}}
\end{align*}
vorgenommen, um einen Vergleich zwischen Messwerten und theoretischen
Berechungnen zu ermöglichen.
Der Skalierungsfaktor wird dabei aus der Startlänge $d_0$ der jeweiligen
Startposition der Spiegel $r_1, r_2$ und der Formel \ref{eqn:c} wir folgt
berechnet:
\begin{align*}
  c = g_1g_2(d_0,r_1,r_2)
\end{align*}

\subsection{Konfokale Konfiguration}
In Tabelle \ref{tab:kk} sind die zur Untersuchung dieser Konfiguration
aufgenommenen Messwerte aufgelistet. In Abbildung \ref{plt:kk} sind diese
graphisch dargestellt. Der Umskalierungsfaktor liegt bei $c = \num{0,43}$. Die zu
sehende Ausgleichskurve wird mit der Formel
\begin{align*}
  f(d) = a\cdot d^2 + b\cdot d + c
\end{align*}
berechnet. Die dabei errechneten Parameter ergeben
\begin{align*}
  %a &= \SI{}{\raiseto{-2}\milli\meter} \\
  %b &= \SI{}{\raiseto{-1}\milli\meter} \\
  %c &= \num{}
\end{align*}

\begin{figure}[htb]
  \centering
  \includegraphics[width=0.7\textwidth]{build/stab-rund.pdf}
  \caption{Gemessene Werte der Untersuchung der Stabilitätsmessung der konfokalen Konfiguration. }
  \label{plt:kk}
\end{figure}
\FloatBarrier

\begin{table}
\centering
\caption{Aufgenommene Messwerte zur Untersuchung der $T_{00}$ Mode. Dabei ist der
Abstand zum Hauptmaxima, sowie die bei dem Anstand vermessene Lichtntensität aufgelistet.}
\begin{tabular}{cccc}
\toprule
Abstand $d$ / $\si{\centi\meter}$ & Intensität $I$ / $\si{\nano\ampere}$  & Abstand $d$ / $\si{\centi\meter}$ & Intensität $I$ / $\si{\nano\ampere}$ \\
\midrule
-28.0 & 5.3 & 2.0 & 3290.0 \\
-26.0 & 3.3 & 4.0 & 3000.0 \\
-24.0 & 12.9 & 6.0 & 2800.0 \\
-22.0 & 26.0 & 8.0 & 2150.0 \\
-20.0 & 32.0 & 10.0 & 1610.0 \\
-18.0 & 73.0 & 12.0 & 1150.0 \\
-16.0 & 148.0 & 14.0 & 820.0 \\
-14.0 & 275.0 & 16.0 & 470.0 \\
-12.0 & 430.0 & 18.0 & 270.0 \\
-10.0 & 815.0 & 20.0 & 140.0 \\
-8.0 & 1090.0 & 22.0 & 40.0 \\
-6.0 & 1470.0 & 24.0 & 35.0 \\
-4.0 & 2180.0 & 26.0 & 11.0 \\
-2.0 & 2630.0 & 28.0 & 9.0 \\
\bottomrule
\end{tabular}
\label{tab:t00}
\end{table}

\FloatBarrier

\subsubsection{Konkav-Ebene Konfiguration}
Die für die Untersuchung der Stabilitätsbedingung aufgenommenen Werte der
Konkav-Ebenen Konfiguration sind in Tabelle \ref{tab:ke} aufgelistet und
in Abbildung \ref{plt:ke} zusätzlich mit der durch
\begin{align*}
  g(d) = m\cdot d + a
\end{align*}
berechneten Ausgleichsgeraden abgebildet. Als Skalierungsfaktor ergibt
sich $c = \num{0.34}$.

\begin{figure}[htb]
  \centering
  \includegraphics[width=0.7\textwidth]{build/stab-flach.pdf}
  \caption{Graphische Darstellung der Messwerte zur Untersuchung der Stabilitätsbedingung bei einer konkav-ebenen Konfiguration.}
  \label{plt:ke}
\end{figure}
\FloatBarrier

\begin{table}
\begin{tabular}{cccc}
col0 & col1 & col2 & col3 \\
-28.0 & 1.5 & 2.0 & 2.5 \\
-26.0 & 5.5 & 4.0 & 42.0 \\
-24.0 & 9.5 & 6.0 & 113.0 \\
-22.0 & 14.0 & 8.0 & 180.0 \\
-20.0 & 25.0 & 10.0 & 260.0 \\
-18.0 & 66.0 & 12.0 & 310.0 \\
-16.0 & 115.0 & 14.0 & 310.0 \\
-14.0 & 160.0 & 16.0 & 240.0 \\
-12.0 & 190.0 & 18.0 & 200.0 \\
-10.0 & 230.0 & 20.0 & 125.0 \\
-8.0 & 200.0 & 22.0 & 62.0 \\
-6.0 & 190.0 & 24.0 & 28.0 \\
-4.0 & 150.0 & 26.0 & 18.0 \\
-2.0 & 70.0 & 28.0 & 14.0 \\
\end{tabular}
\end{table}

\FloatBarrier
