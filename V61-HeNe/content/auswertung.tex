\newpage
\section{Auswertung}

\label{sec:Auswertung}
Im Folgenden wurden alle Ausgleichsrechnungen mit der Funktion \textit{curve\_fit} von dem
Python Paket \textit{scipy.optimize} \cite{scipy} durchgeführt, ebenso werden deren
Parameter mit diesem Paket. Die Fehlerberechnung wird mit dem Paket \textit{uncertainties}
\cite{uncertainties} von Python durchgeführt. Die Umrechnung von
Einheiten, sowie weitere Berechnungen wurden mit dem \textit{numpy} Paket \cite{numpy}
durchgeführt.

\subsection{Bestimmung der Wellenlänge}
Die zur Bestimmung der Wellenlänge des Lasers aufgenommenen Abstände der Maxima
zum Hauptmaximum
sind in Tabelle \ref{tab:welle} dargestellt.
Dabei wurde der Laser in einem Abstand
von $L = \SI{11.3(2)}{\centi\meter}$ positioniert und ein Gitter mit einem
Abstand der einzelnen Streben von $a = \SI{0.1}{\milli\meter}$ verwendet.
Mithilfe der Formel \ref{eqn:welle} wurden die Wellenlängenberechnungen an den
einzelnen Intensitätsmaxima durchgeführt. Die daraus resultierenden Werte für
$\lambda$ sind in Tabelle \ref{tab:welle} nachzulesen.

\begin{table}
\centering
\begin{tabular}{ccc}
\toprule
Ordnung $n$ & Abstand $d$ / $\si{meter}$ & Wellenlänge $\lambda$ / $\si{\nano\meter}$ \\
\midrule
1.0 & 0.068 & 515.61 \\
2.0 & 0.145 & 438.92 \\
3.0 & 0.214 & 353.82 \\
4.0 & 0.29 & 295.34 \\
5.0 & 0.376 & 252.99 \\
-1.0 & 0.079 & 572.98 \\
-2.0 & 0.14 & 431.23 \\
-3.0 & 0.212 & 352.61 \\
-4.0 & 0.288 & 294.78 \\
-5.0 & 0.367 & 251.69 \\
\bottomrule
\end{tabular}
\caption{Aufgenommene Messwerte zur Bestimmung der Wellenlänge des verwendeten Lasers.}
\label{tab:welle}
\end{table}


Dabei beschreibt die Ordnung $n>0$ die auftretenden Maxima zur rechten Seite
senkrecht zur optischen Achse und $n<0$ die Maxima zur linken Seite.
Eine Mittelung über alle berechneten Wellenlängen ergab
\begin{align*}
  \bar{\lambda} = \SI{375.99(10489)}{\nano\meter}.
\end{align*}

\newpage

\subsection{Untersuchung der TEM-Moden}

\subsubsection{Auswertung der $T_{00}$ Mode}
Die in Abbildung \ref{plt:t00} dargestellten Messwerte für die Intenstät $I$
und die Entfernung vom Hauptmaxima $d$ sind in Tabelle \ref{tab:t00}
aufgelistet. Aus der Messung des Abstandes mit einem Maßband resultiert eine
Unsicherheit von $\pm\SI{0.5}{\centi\meter}$.

\begin{table}
\centering
\caption{Aufgenommene Messwerte zur Untersuchung der $T_{00}$ Mode. Dabei ist der
Abstand zum Hauptmaxima, sowie die bei dem Anstand vermessene Lichtntensität aufgelistet.}
\begin{tabular}{cccc}
\toprule
Abstand $d$ / $\si{\centi\meter}$ & Intensität $I$ / $\si{\nano\ampere}$  & Abstand $d$ / $\si{\centi\meter}$ & Intensität $I$ / $\si{\nano\ampere}$ \\
\midrule
-28.0 & 5.3 & 2.0 & 3290.0 \\
-26.0 & 3.3 & 4.0 & 3000.0 \\
-24.0 & 12.9 & 6.0 & 2800.0 \\
-22.0 & 26.0 & 8.0 & 2150.0 \\
-20.0 & 32.0 & 10.0 & 1610.0 \\
-18.0 & 73.0 & 12.0 & 1150.0 \\
-16.0 & 148.0 & 14.0 & 820.0 \\
-14.0 & 275.0 & 16.0 & 470.0 \\
-12.0 & 430.0 & 18.0 & 270.0 \\
-10.0 & 815.0 & 20.0 & 140.0 \\
-8.0 & 1090.0 & 22.0 & 40.0 \\
-6.0 & 1470.0 & 24.0 & 35.0 \\
-4.0 & 2180.0 & 26.0 & 11.0 \\
-2.0 & 2630.0 & 28.0 & 9.0 \\
\bottomrule
\end{tabular}
\label{tab:t00}
\end{table}


\begin{figure}[htb]
  \centering
  \includegraphics[width=0.7\textwidth]{T00.pdf}
  \caption{Graphische Darstellung der Werte aus Tabelle \ref{tab:t00} mit Ausgleichskurve der Form von Formel \ref{eqn:t00}.}
  \label{plt:t00}
\end{figure}

Die durch eine Funktion der Form von Formel \ref{eqn:t00} erzeugte Ausgleichsrechung berechneten Parameter
ergeben
\begin{align*}
  I_0 &= \SI{2.09(8)}{\nano\ampere} \\
  d_0 &= \SI{3212(31)}{\milli\meter} \\
  \omega &= \SI{13.87(16)}{\raiseto{-1}\milli\meter}.
\end{align*}
\FloatBarrier

\subsubsection{Auswertung der $T_{10}$ Mode}

Die für die Untersuchung der $T_{10}$ Mode aufgenommen Intensitäten $I$,
sowie Abstände $d$ sind in Tabelle \ref{tab:t10} aufgelistet und in
Abbildung \ref{plt:t10} zusammen mit einer berechneten Ausgleichskurve durch
Formel \ref{eqn:t10} graphisch dargestellt.

\begin{table}
\begin{tabular}{cccc}
col0 & col1 & col2 & col3 \\
-28.0 & 1.5 & 2.0 & 2.5 \\
-26.0 & 5.5 & 4.0 & 42.0 \\
-24.0 & 9.5 & 6.0 & 113.0 \\
-22.0 & 14.0 & 8.0 & 180.0 \\
-20.0 & 25.0 & 10.0 & 260.0 \\
-18.0 & 66.0 & 12.0 & 310.0 \\
-16.0 & 115.0 & 14.0 & 310.0 \\
-14.0 & 160.0 & 16.0 & 240.0 \\
-12.0 & 190.0 & 18.0 & 200.0 \\
-10.0 & 230.0 & 20.0 & 125.0 \\
-8.0 & 200.0 & 22.0 & 62.0 \\
-6.0 & 190.0 & 24.0 & 28.0 \\
-4.0 & 150.0 & 26.0 & 18.0 \\
-2.0 & 70.0 & 28.0 & 14.0 \\
\end{tabular}
\end{table}


\begin{figure}[htb]
  \centering
  \includegraphics[width=0.7\textwidth]{T10.pdf}
  \caption{Darstellung der in Tabelle \ref{tab:t10} aufgelisteten Messwerte und einer berechneten Ausgleichskurve der Form von Formel \ref{eqn:t10}.}
  \label{plt:t10}
\end{figure}

Eine Regression mit einer Funktion der Form von Formel \ref{eqn:t10} ergab die Parameter
\begin{align*}
  I_{0,1} &= \SI{227(8)}{\nano\ampere} \ \ \ \ \ \ \ \ \ \ \ \ \ I_{0,2} = \SI{315(8)}{\nano\ampere} \\
  d_{0,1} &= \SI{-9.66(21)}{\milli\meter} \, \ \ \ \ d_{0,2} = \SI{13.19(14)}{\milli\meter} \\
  \omega_1 &= \SI{10.2(4)}{\raiseto{-1}\milli\meter} \ \ \ \ \ \ \ \omega_2 = \SI{9.60(29)}{\raiseto{-1}\milli\meter}
\end{align*}
\FloatBarrier

\subsection{Untersuchung der Polarisation}
Die Tabelle \ref{tab:pol} enthält alle zur Untersuchung der Polarisation
des Lasers verwendeten Messdaten. Dabei ist der Winkel sowohl in Grad als
auch in Rad angegeben.
Die durch die Ausgleichsrechnung durch eine Funktion der Formel \ref{eqn:pol}
bestimmten Parameter ergaben sich zu
\begin{align*}
  I_0 &= \SI{2.83(6)}{\micro\ampere} \\
  \phi_0 &= \SI{-87.72(2)}{\radian} \\
      &= \SI{-13.96(0)}{\degree}.
\end{align*}
In Abbildung \ref{plt:pol} sind die aufgenommenen Messwerte der Intensität $I$ samt
Regression gegen den Drehwinkel $\phi$ des Polarisationsfilters dargestellt.

\begin{figure}[htb]
  \centering
  \includegraphics[width=0.7\textwidth]{build/polarisation.pdf}
  \caption{Graphische Abbildung der Messdaten aus Tabelle \ref{tab:pol} mit zugehöriger Ausgleichskurve durch eine Funktion der Formel \ref{eqn:pol}.}
  \label{plt:pol}
\end{figure}

\begin{table}
\centering
\caption{Aufgenommene Messwerte zur Untersuchung der Polarisation des Lichtstrahls.
Aufgelistet sind der Winkel $\phi$, in Grad, sowie in Radiant. Zudem ist die bei dem
Winkel auftretende Lichtintensität nachzulesen.}
\label{tab:pol}
\begin{tabular}{cccccc}
\toprule
$\phi$ / $\si{\degree}$ & $\phi$ / $\si{\radian}$ & $I$ / $\si{\nano\ampere}$ & $\phi$ / $\si{\degree}$ & $\phi$ / $\si{\radian}$ & $I$ / $\si{\nano\ampere}$ \\
\midrule
10.0 & 0.17 & 508.0 & 190.0 & 0.17 & 508.0 \\
20.0 & 0.35 & 910.0 & 200.0 & 0.35 & 910.0 \\
30.0 & 0.52 & 1370.0 & 210.0 & 0.52 & 1370.0 \\
40.0 & 0.7 & 1760.0 & 220.0 & 0.7 & 1760.0 \\
50.0 & 0.87 & 2680.0 & 230.0 & 0.87 & 2680.0 \\
60.0 & 1.05 & 2800.0 & 240.0 & 1.05 & 2800.0 \\
70.0 & 1.22 & 3300.0 & 250.0 & 1.22 & 3300.0 \\
80.0 & 1.4 & 3180.0 & 260.0 & 1.4 & 3180.0 \\
90.0 & 1.57 & 3100.0 & 270.0 & 1.57 & 3100.0 \\
100.0 & 1.75 & 2420.0 & 280.0 & 1.75 & 2420.0 \\
110.0 & 1.92 & 1720.0 & 290.0 & 1.92 & 1720.0 \\
120.0 & 2.09 & 1220.0 & 300.0 & 2.09 & 1220.0 \\
130.0 & 2.27 & 780.0 & 310.0 & 2.27 & 780.0 \\
140.0 & 2.44 & 376.0 & 320.0 & 2.44 & 376.0 \\
150.0 & 2.62 & 127.0 & 330.0 & 2.62 & 127.0 \\
160.0 & 2.79 & 11.2 & 340.0 & 2.79 & 11.2 \\
170.0 & 2.97 & 29.7 & 350.0 & 2.97 & 29.7 \\
\bottomrule
\end{tabular}
\end{table}

\FloatBarrier

\subsection{Überprüfung der Stabilitätsbedingung}
Bei der Untersuchung der Stabilitätsbedingung werden zwei verschiedene
Konfigurationen untersucht. Dabei werden bei der ersten Untersuchung zwei
konfokale Spiegel verwendet, bei der zweiten hingegen ein konfokaler und
ein ebener Spiegel. Dabei wird eine Umskalierung der Messwerte der Art
\begin{equation*}
  I \rightarrow \frac{I \cdot c}{I_\text{max}}
\end{equation*}
mit der maximal gemessenen Intensität $I_\text{max}$ vorgenommen, um einen
Vergleich zwischen Messwerten und theoretischen
Berechungnen zu ermöglichen.
Der Skalierungsfaktor $c$ wird dabei aus der Startlänge $d_0$ der jeweiligen
Startposition der Spiegel $r_1, r_2$ und der Formel \ref{eqn:c} wir folgt
berechnet:
\begin{equation}
  c = g_1g_2(d_0,r_1,r_2)
  \label{eqn:c}
\end{equation}

\newpage
\subsubsection{Konfokale Konfiguration}
In Tabelle  sind die zur Untersuchung dieser Konfiguration
aufgenommenen Messwerte des Abstandes zur Strahlenachse $d$, die Intensität $I$
und die Umskalierte Intensität $\frac{I \cdot c}{I_\text{max}}$ aufgelistet. In
Abbildung \ref{plt:kk} sind diese
graphisch dargestellt. Der Umskalierungsfaktor liegt nach Formel \ref{eqn:c} bei
$c = \num{0,43}$. Die zu sehende Ausgleichskurve wird mit einer Funktion der Form
\begin{align*}
  f(d) = a\cdot d^2 + b\cdot d + c
\end{align*}
berechnet. Die dabei errechneten Parameter ergeben
\begin{align*}
  a &= \SI{2.5(21)e-5}{\raiseto{-2}\milli\meter} \\
  b &= \SI{-2(4)e-2}{\raiseto{-1}\milli\meter} \\
  c &= \num{2(20)e-2}
\end{align*}

\begin{figure}[htb]
  \centering
  \includegraphics[width=0.6\textwidth]{build/stab-rund.pdf}
  \caption{Gemessene Werte der Untersuchung der Stabilitätsmessung der konfokalen Konfiguration. }
  \label{plt:kk}
\end{figure}

\begin{table}
\centering
\caption{Auflistung der aufgenommenen Messwerte zur Überprüfung der
Stabilitätsbedingung bei einer konfokalen Spiegelanordnung. Dabei beschreibt
$d$ den Abstand senkrecht zur Strahlenachse, $I$ die dort aufzufindene Intensität
und $\frac{I \cdot c}{I_\text{max}}$ den Skalierten Wert der Intensität.}

\begin{tabular}{ccc}
\toprule
$d$ / \si{\centi\meter} & $I$ / \si{\nano\ampere} & $\frac{I \cdot c}{I_\text{max}}$ / \si{\nano\ampere}\\
\midrule
47.3 & 1.85 & 0.01 \\
52.5 & 1.71 & 0.01 \\
64.5 & 2.04 & 0.01 \\
78.2 & 10.3 & 0.04 \\
94.5 & 9.6 & 0.04 \\
112.6 & 16.3 & 0.07 \\
124.7 & 46.5 & 0.19 \\
140.0 & 96.6 & 0.39 \\
150.0 & 108.5 & 0.44 \\
158.3 & 63.2 & 0.26 \\
161.1 & 85.3 & 0.34 \\
\bottomrule
\end{tabular}
\label{tab:kk}
\end{table}

\FloatBarrier

\subsubsection{Konkav-Ebene Konfiguration}
Die für die Untersuchung der Stabilitätsbedingung aufgenommenen Werte der
Konkav-Ebenen Konfiguration sind in Tabelle \ref{tab:ke} aufgelistet.
In Abbildung \ref{plt:ke} sind zusätzlich mit einer Formel der Form
\begin{align*}
  g(d) = m\cdot d + a
\end{align*}
berechneten Ausgleichsgeraden abgebildet. Als Skalierungsfaktor ergibt
sich $c = \num{0.34}$. Aus dieser Ausgleichsrechung ergeben sich die Parameter
\begin{align*}
   m &= \SI{1.02(22)}{\raiseto{-1}\milli\meter} \\
   a &= \num{-0.79(22)}
\end{align*}

\begin{figure}[htb]
  \centering
  \includegraphics[width=0.7\textwidth]{build/stab-flach.pdf}
  \caption{Graphische Darstellung der Messwerte zur Untersuchung der Stabilitätsbedingung bei einer konkav-ebenen Konfiguration.}
  \label{plt:ke}
\end{figure}

\begin{table}
\centering
\caption{Auflistung der aufgenommenen Messwerte zur Überprüfung der
Stabilitätsbedingung bei einer konkav-ebenen Spiegelanordnung. Dabei beschreibt
$d$ den Abstand senkrecht zur Strahlenachse, $I$ die dort aufzufindene Intensität
und $\frac{I \cdot c}{I_\text{max}}$ den Skalierten Wert der Intensität.}
\begin{tabular}{ccc}
\toprule
$d$ / \si{\centi\meter} & $I$ / \si{\nano\ampere} & $\frac{I \cdot c}{I_\text{max}}$ / \si{\nano\ampere}\\
\midrule
92.3 & 0.38 & 0.12 \\
96.7 & 0.71 & 0.22 \\
100.8 & 0.82 & 0.25 \\
112.5 & 1.1 & 0.34 \\
\bottomrule
\end{tabular}
\label{tab:ke}
\end{table}

\FloatBarrier
