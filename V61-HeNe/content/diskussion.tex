\newpage
\section{Diskussion}
Die Ergebnisse der Messungen hängen stark von der Zusammensetzung der umgebenden
Luft und der Sauberkeit der Spiegel und Linsen ab. Staub und Abdrücke verfälschen
daher die aufgenommenen Werte durch Schwankung der Intensitäten. Eine
Abschirmung von Streulicht, sowie ein Laser umschließender Aufbau würde diese
Fehlerquellen gering halten.

Eine weitere Fehlerquelle stellt die Umstellung des Messbereichs des
Amperemeters innerhalb einer dar, welches an die Photodiode angeschlossen ist.
Dabei wird der Fehler größer, je kleiner die zu messende Intensität ist.

Bei Betrachtung der Berechnung der Wellenlänge tritt eine Abweichung zum
Theoriewert ($\lambda = \SI{632}{\nano\meter}$ \cite{anleitung}) von
$\SI{40}{\percent}$ auf. Wie in der Tabelle \ref{tab:welle} zu erkennen,
wird die Abweichung bei Maxima höherer Ordnung immer größer. Dies tritt sowohl
bei der Messung in die eine, als auch in die andere Richtung auf. Daraus ist eine
fehlerhafte Messung der Entfernung zu schlussfolgern. Zudem ist der Spiegel, an
dem die Messung durchgeführt wurde, per Hand senkrecht auf die Strahlenachse
gerichtet, was einer gewisse Ungenauigkeit mit sich bringt. Eine bessere Genauigkeit könnte
mit der Justage dieses Schirms durch eine mechanische Apparatur gewährleistet
werden.

Die Untersuchung der TEM-Moden zeigt bei Betrachtung der $T_{00}$ Mode, sowie
der $T_{10}$ Mode eine gute Übereinstimmung zwischen gemessenen Werten und
Ausgleichsgeraden. In Abbildung \ref{plt:t10} ist ein erhöhtes Minimum der
Ausgleichskurve zu sehen, welches durch Effekte bei der numerischen Berechnung
dieser Kurve zu erklären ist.

Die Betrachtung der Polarisation, wie in Abbildung \ref{plt:pol} zu sehen, weißt
ein erhöhtes Maximum zwischen $\frac{1}{4}\pi$ und $\frac{1}{2}\pi$ auf. Eine
Erklärung dafür ist eine Erhöhung der Intensität durch Streulicht aus der
Umgebung. Zwar ist der Raum abgedunkelt, in dem das Experiment durchgeführt wurde,
allerdings reicht das Streulicht des Lasers selbst, um die
Messergebnisse nach oben zu verfälschen.

Bei Untersuchung der Stabilitätsbedingung ist eine grundlegende
Abweichung zu den erwarteten Theoriekurven zu erkennen.
Dies ist durch falsche Justage zu erklären.
Daher bietet dieser Abschnitt keine Möglichkeit, Aussagen über die
Stabilitätsbedingungen bei der bikonkaven, sowie der konkav-ebenen Konfiguration
zu treffen. Für aussagkräftige Ergebnisse müsste das Experiment
wiederholt werden.
